\documentclass{siamltex}

\usepackage{ucs}
\usepackage[utf8x]{inputenc}
\usepackage{amsmath}
\usepackage{cite}

\usepackage{fontenc}
\usepackage[pdftex]{graphicx}

\usepackage[pdftex,
            pdfauthor={Wolfgang Bangerth, Timo Heister, Luca Heltai, Guido Kanschat, Martin Kronbichler, Matthias Maier, Toby D. Young},
            pdftitle={The deal.II Library, Version 8.0, 2013},
            colorlinks,linkcolor=blue,urlcolor=blue,citecolor=blue]{hyperref}


\newcommand{\specialword}[1]{\texttt{#1}}            
\newcommand{\dealii}{{\specialword{deal.II}}}
\newcommand{\pfrst}{{\specialword{p4est}}}
\newcommand{\trilinos}{{\specialword{Trilinos}}}
\newcommand{\aspect}{\specialword{Aspect}}
\newcommand{\petsc}{\specialword{PETSc}}
\newcommand{\cmake}{{\specialword{CMake}}}
\newcommand{\autoconf}{{\specialword{autoconf}}}

\title{The \dealii{} Library, Version 8.0}
\author{
  Wolfgang Bangerth%
  \thanks{Department of Mathematics, Texas A\&M University, College Station, 
    TX 77843, USA,
    \href{mailto:bangerth@math.tamu.edu}{\texttt{bangerth@math.tamu.edu}}}
  \and
  Timo Heister%
  \thanks{Mathematical Sciences,
O-110 Martin Hall.
Clemson University.
Clemson, SC 29634, USA,
    \href{mailto:heister@clemson.edu}{\texttt{heister@clemson.edu}}}
  \and
  Luca Heltai%
  \thanks{SISSA - International School for Advanced Studies, Via
    Bonomea 265, 34136 Trieste, Italy,
    \href{mailto:luca.heltai@sissa.it}{\texttt{luca.heltai@sissa.it}}}
  \and
  Guido Kanschat%
  \thanks{Interdisciplinary Center for Scientific Computing, Heidelberg
          University, Im Neuenheimer Feld 368, 69120 Heidelberg, Germany,
          \href{mailto:kanschat@uni-heidelberg.de}{\texttt{kanschat@uni-heidelberg.de}}}
  \and
  Martin Kronbichler%
  \thanks{Institute for Computational Mechanics, Technische Universit\"at M\"unchen, Boltzmannstr.~15, 85748 Garching b. M\"unchen, Germany, \href{mailto:kronbichler@lnm.mw.tum.de}{\texttt{kronbichler@lnm.mw.tum.de}}}
  \and
  Matthias Maier%
  \thanks{Institute of Applied Mathematics, Heidelberg
          University, Im Neuenheimer Feld 293/294, 69120 Heidelberg, Germany,
          \href{mailto:matthias.maier@iwr.uni-heidelberg.de}{\texttt{matthias.maier@iwr.uni-heidelberg.de}}}
  \and
  Bruno Turcksin%
  \thanks{Department of Mathematics, Texas A\&M University, College Station, 
    TX 77843, USA,
    \href{mailto:turcksin@math.tamu.edu}{\texttt{turcksin@math.tamu.edu}}}
  \and
  Toby~D.~Young%
  \thanks{Institute of Fundamental Technological Research of the Polish Academy of Sciences, ul. Pawi\'nskiego 5b, Warsaw 02-106, Poland,
    \href{mailto:tyoung@ippt.pan.pl}{\texttt{tyoung@ippt.pan.pl}}}
}

\renewcommand{\labelitemi}{--}


\begin{document}
\maketitle

\begin{abstract}
  This paper provides an overview of the new features of the finite element library \dealii{} version 8.0.
\end{abstract}


\section{Overview}

\dealii{} version 8.0 was released July 24, 2013. This paper provides an
overview of the new features of this release and serves as a citable web
reference for the \dealii{} software library version 8.0. \dealii{} is an
object-oriented finite element library used around the world in the
development of finite element solvers. It is available for free under the
GNU Lesser General Public License (LGPL) from the \dealii{} homepage at
\url{http://www.dealii.org/}.

Version 8.0 is a major release. It has numerous significant features along
with the usual set of bug fixes and documentation updates. In particular,
it has the following noteworthy large changes:
\begin{itemize}
  \item The configuration and build system has been switched to \cmake{},
    providing better support for a wide variety of platforms, better
    integration with IDEs such as Eclipse, and many other advantages. See Section \ref{sec:cmake}.

\item \dealii{} now supports 64-bit integers for degrees of freedom indices for
  problems with more than 2 billion unknowns and has been tested on problems
  of up to 27 billion unknowns. See Section \ref{sec:64bit}.

\item \dealii{} is now licensed under the GNU Lesser General Public License
  version 2.1 or later (LGPL-2.1+), see Section \ref{sec:license} for details.

\item There is the usual set of dozens or hundreds of new small features and
  bugfixes, some of which are described in \ref{sec:other}.
\end{itemize}
Information on how to cite \dealii{} is provided in Section \ref{sec:cite}.

\section{Important Changes}

\subsection{Build System}\label{sec:cmake}

With the release of version 8.0 and after more than 12 years with an
\autoconf{} and custom makefiles setup, \dealii{} switched its build system
over to \cmake{}. This is a major rewrite which changes over $25,000$ lines
of code (10,000 insertions and 15,000 deletions) together with a major
reorganization and cleanup of the code base. The rewrite addressed quite a
number of shortcomings with the old build system, that made a rewrite
highly desirable (if not necessary):
\begin{itemize}
  \item Over time, with the addition of an increasing number of external
    libraries, \dealii{} can optionally interface with, the configuration
    happened to be highly heterogeneous---almost every external dependency
    had to be set up with different command line options with completely
    different internal setup logic in the \autoconf{} files.
  \item Nowadays, \dealii{} is used in highly different environments
    ranging from installations on laptops to computing clusters. The old
    build system was not suited very well for quite a number of them,
    e.\,g. distributional needs or static linkage.
  \item The old build system strongly depended on the availability of a
    \textsc{UNIX} or \textsc{GNU/Linux} environment due to the fact that
    \textsc{GNU Make}, \textsc{Bash} and \text{Perl} were mandatory build
    time dependencies,
    which made support for other platforms (and especially Microsoft
    Windows) difficult.
\end{itemize}
Arguably, it would have been possible to address all this shortcomings in
the current build system, or by changing to another alternative, but there
are a number of reasons that make \cmake{} a particularly good choice as
build system for \dealii{}:
\begin{itemize}
  \item \cmake{} readily provides elaborate support for dependency
    resolution, incremental rebuilds and parallel builds, which were
    previously implemented mainly by hand.
  \item \cmake{} is a meta build system. It provides support for
    \emph{native toolchains} for \textsc{Linux}, \textsc{BSD},
    \textsc{Darwin}, and \textsc{Windows} platforms with minimal external
    dependencies, which drastically increases portability.
  \item \cmake{} supports a variety of commonly used IDEs for different
    platforms such as \textsc{Eclipse}, \textsc{KDevelop}, \textsc{Xcode},
    \textsc{CodeBlocks} or \textsc{MSVC} where it can directly produce a
    corresponding project.
\end{itemize}
The fact that the new build system was rewritten from scratch, offered the
possibility to reformulate and respect some major design criteria. Amongst
them are
\begin{itemize}
  \item Full support for all 16 external dependencies in various
    setups---either compiled and/or installed by the user, or provided by
    software distributions such as Linux distributions or Mac Ports.
  \item Feature auto detection to ease the setup for the average user,
    while at the same time providing full user override of configuration
    options to support almost any setup case.
  \item Be (almost) readily packageable for major distributions (e.\,g.
    Debian Linux, or Ubuntu).
  \item Provide a project configuration so that user projects based on
    \dealii{} only need a minimalistic \textsc{CMakeLists.txt} file on
    client side.
\end{itemize}

\subsection{64-bit Indices}\label{sec:64bit}

Up to version 7.3 of \dealii{}, the indices of degrees of freedom (which
are uniquely defined) were stored using the \emph{unsigned int} datatype. This
limited the total number of degrees of freedom in a computation to about four billion. When \dealii{}
was used in conjunction with \petsc{} or \trilinos{}, this number was in
fact only half as large since both of these libraries use \emph{signed
integer} indices (leaving only 31 bits for the indices). Furthermore, because an individual processor needs at
least on the order of $10^5$ unknowns to have a suitable amount of work,
\dealii{} could not scale to more than about $\frac{2\cdot 10^9}{10^5}=20,000$
processors. Even though this number may seem large, an increasing number of
supercomputers offer a much larger number of processors -- the fastest
supercomputers today are almost one hundred times larger. It is also
important to note that using simulations with more than two billions of
unknowns can be run using only a couple of thousand processors. This number
is easily reached by clusters at typical universities in the West.

To increase the number of unknowns that can be solved using \dealii{}, the
degrees of freedom can now be stored using \emph{unsigned long long int} (a
data type typically 64 bit wide) and the new 64 bit capabilities of Epetra
(Trilinos 11.2 and later) are employed (PETSc could already be compiled to use 64
bit indices for a long time). The new version of \dealii{} is completely
backward compatible with the previous version and can be compiled to use 32
or 64 bit indices indifferently. This was done by using a typedef
\emph{types::global\_dof\_index} to switch between \emph{unsigned int} and
\emph{unsigned long long int} indices. The switch between 32 and 64 bit
indices is done using the cmake option \hbox{\tt
  -DDEAL\_II\_WITH\_64BIT\_INDICES=OFF/ON}. By default, this option is turned
off to keep memory consumption at moderate levels. 

Using 64 bit indices only makes sense for parallel computations for which
\dealii{} relies on either \petsc{} or \trilinos{}.  To use the 64 bit indices
with the former, \petsc{} must be configured with {\tt
  --with-64-bit-indices}. For the latter, \trilinos{} 11.2 or later must be
used (by the default \trilinos{} is compiled with 32 and 64 bit capabilities) and
uses the facilities laid out in \cite{trilinos64}. 

To write a code compatible for 32 and 64 bit indices,
\emph{types::global\_dof\_index} needs to be used in all places where the
global indices of degrees of freedom are referenced. There are of course many
of these places: the library contained some 20,000 occurrences of the text
\texttt{unsigned int}, all of which needed to be examined, and some 1,600
places had to be modified. However, the library now compiles and runs with 64
bit indices and the adapted version of the step-40 tutorial program was run on
3600 processors; this computation had 1.47 billions of active cells, 5.898
billions of degrees of freedom, and was solved in 771 seconds.




\subsection{License Change}\label{sec:license}

\dealii{} used to be distributed under the Q Public License (QPL), a
license that was popular in the late 1990s mainly due to its utilization by
the Qt Application Framework.

The QPL, as many other open source licenses, puts restrictions on any use;
in particular, programs that are developed to use \dealii{} needed to be
open source as well. In practice, this puts some restrictions on commercial
use because companies are unwilling to engage in a model where they are
required to deliver their source code to customers who may then distribute
it further.

This license was chosen in the early days of \dealii{} out of a feeling that
whoever uses the library should be required to show the same open source
spirit. Abstractly, many will probably agree with this sentiment. However, in
more than 10 years of reality, it has only led to companies walking away from
using \dealii{} for commercial products. As a consequence, the project has
seen no gain, only losses from this choice of license. 

We have therefore decided to change the license under which \dealii{} is
distributed to the GNU Lesser General Public License (LGPL), version 2.1 or
later at the discretion of the user. Among the advantages of this choice
are:
\begin{itemize}
  \item The LGPL allows proprietary software to dynamically link against
    \dealii{} without the need to redistribute this software under a
    restrictive license. On the other hand, the LGPL allows us to leave the
    full ``copyleft'' of the library itself intact.
  \item It avoids licensing incompatibilities with (external) dependencies
    that are either licensed under (L)GPL or a BSD style license---which
    would otherwise make any binary distribution of \dealii{} impossible,
    for example for inclusion in the package management system of many
    operating systems. This is especially important given the large number
    of external packages \dealii{} interacts with these days.
\end{itemize}
A further account of these considerations can be found in \cite{BH13}.


\subsection{Other Changes}\label{sec:other}

The \dealii{} release 8.0 also includes improvements in the following areas:

\begin{itemize}
\item Improvements to \dealii{} threading support: The WorkStream class used
  in \dealii{} to parallelize assembly loops now uses thread local memory,
  which gives considerably better utilization of caches on multi-core
  architectures. In addition, a few more linear algebra functions have been
  parallelized with threads. The new version also allows to directly limit the
  number of threads used by the task scheduler in Threading Building Blocks at
  run time.
\item \dealii{}'s own iterative solvers have been revised. CG and BiCGStab now
  use less auxiliary vectors, and GMRES avoids unnecessary work due to
  re-orthogonalization when it is not necessary. This makes behavior more
  similar to external linear algebra in PETSc and Trilinos (but those remain
  better adapted to large-scale parallel computers by reducing global
  communication).
\item The interfaces to parallel linear algebra objects of PETSc and Trilinos
  objects have been further unified to allow for easily exchanging one variant
  with the other.
\item The generic template instantiation mechanism in \dealii{} has been
  applied to several new places to pre-compile code for all vector types in
  ErrorEstimator, VectorTools, FETools, ConstraintMatrix.
\end{itemize}


\section{How to cite \dealii{}}\label{sec:cite}

In order to justify the work the developers of \dealii{} put into this
software, we ask that papers using the library reference one of the \dealii{}
papers. This helps us justify the effort we put into it.

There are various ways to reference \dealii{}. To acknowledge the use of a
particular version of the library, reference the present document as follows:
\begin{verbatim}
@article{dealII80,
  title = {The {\tt deal.{I}{I}} Library, Version 8.0},
  author = {W. Bangerth and T. Heister and L. Heltai and G. Kanschat
   and M. Kronbichler and M. Maier and B. Turcksin and T. D. Young},
  journal = {arXiv preprint \url{http://arxiv.org/abs/1312.2266v3}},
  year = {2013},
}
\end{verbatim}

The original \texttt{\dealii{}} paper containing an overview of its
architecture is \cite{BangerthHartmannKanschat2007}. If you rely on specific
features of the library, please consider citing any of the following:
\begin{itemize}
 \item For geometric multigrid: \cite{Kanschat2004,JanssenKanschat2011};
 \item For distributed parallel computing: \cite{BangerthBursteddeHeisterKronbichler11};
 \item For $hp$ adaptivity: \cite{BangerthKayserHerold2007};
 \item For matrix-free and fast assembly techniques:
   \cite{KronbichlerKormann2012};
 \item For computations on lower-dimensional manifolds:
   \cite{DeSimoneHeltaiManigrasso2009}.
\end{itemize}


\nocite{BangerthKanschat1999}

\section{Acknowledgements}

\dealii{} is a world-wide project with dozens of contributors around the
globe. Other than the authors of this paper, the following people contributed code to
this release: 
  Fahad Alrashed,
  Juan Carlos Araujo Cabarcas,
  Denis Davydov,
  Ashkan Dorostkar,
  J{\"o}rg Frohne,
  Felix Gruber,
  Scott Miller,
  Spencer Patty,
  Martin Steigemann,
  Kainan Wang,
  Christian W{\"u}lker.
Their contributions are much appreciated!

\dealii{} and its developers are financially supported through a
variety of funding sources. W.~Bangerth and B.~Turcksin were partially
supported by the National Science Foundation under award OCI-1148116
as part of the Software Infrastructure for Sustained Innovation (SI2)
program; by the Computational Infrastructure in Geodynamics initiative
(CIG), through the National Science Foundation under Award
No.~EAR-0949446 and The University of California -- Davis; and through
Award No.~KUS-C1-016-04, made by King Abdullah University of Science
and Technology (KAUST). 

L.~Heltai was partially supported by the project OpenSHIP,
``Simulazioni di fluidodinamica computazionale (CFD) di alta qualita
per le previsioni di prestazioni idrodinamiche del sistema
carena-elica in ambiente OpenSOURCE'', financed by Regione FVG - POR
FESR 2007–2013 Obiettivo competitivita regionale e occupazione.

T.~Heister was partially supported by the Computational Infrastructure in
Geodynamics initiative (CIG), through the National Science Foundation under Award
No. EAR-0949446 and The University of California -- Davis and through Award No. KUS-
C1-016-04, made by King Abdullah University of Science and Technology (KAUST).

\bibliography{deal80}{}
\bibliographystyle{plain}

\end{document}

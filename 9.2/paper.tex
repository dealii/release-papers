\documentclass{ansarticle-preprint}
%\usepackage{ucs}
\usepackage[utf8]{inputenc}
\usepackage{amsmath}
%\usepackage{cite}
\usepackage{anslistings}
\usepackage{multicol}
\usepackage{pdfsync}

\usepackage{pgfplots}
\usepackage{pgfplotstable}

\usepackage{fontenc}
\usepackage{graphicx}
\usepackage{xspace}

\usepackage[normalem]{ulem}

\usepackage{todonotes}

\pgfplotsset{compat=1.9}

\newcommand{\specialword}[1]{\texttt{#1}}
\newcommand{\dealii}{{\specialword{deal.II}}\xspace}
\newcommand{\pfrst}{{\specialword{p4est}}\xspace}
\newcommand{\trilinos}{{\specialword{Trilinos}}\xspace}
\newcommand{\aspect}{\specialword{Aspect}\xspace}
\newcommand{\petsc}{\specialword{PETSc}\xspace}
\newcommand{\cmake}{{\specialword{CMake}}\xspace}
\newcommand{\autoconf}{{\specialword{autoconf}}\xspace}

%
% Author list -- please add yourself in both places below (in
%                alphabetical order) if you think that your
%                contributions to the last release warrant this
%

\hypersetup{
  pdfauthor={
     Daniel Arndt,
     Wolfgang Bangerth,
%     Thomas C. Clevenger,
%     Denis Davydov,
%     Marc Fehling,
%     Daniel Garcia-Sanchez,
%     Graham Harper,
    Timo Heister,
%     Luca Heltai,
    Martin Kronbichler,
%     Ross Maguire Kynch,
    Matthias Maier,
%     Jean-Paul Pelteret,
%     Bruno Turcksin,
%     David Wells
  },
  pdftitle={The deal.II Library, Version 9.2, 2020},
}

\title{The \dealii Library, Version 9.2}

 \author[*1]{Daniel Arndt}
 \affil[1]{Computational Engineering and Energy Sciences Group,
   Computional Sciences and Engineering Division,
   Oak Ridge National Laboratory, 1 Bethel Valley Rd.,
   TN 37831, USA.
   \texttt{arndtd@ornl.gov}}

 \author[2]{Wolfgang~Bangerth}
 \affil[2]{Department of Mathematics, Colorado State University, Fort
   Collins, CO 80523-1874, USA.
   \texttt{bangerth@colostate.edu}}

% \author[3]{Thomas~C.~Clevenger}
% \affil[3]{School of Mathematical and Statistical Sciences,
%   Clemson University,
%   Clemson, SC, 29634, USA
%   {\texttt{\{tcleven, heister\}@clemson.edu}}}
%
% \author[4]{Denis~Davydov}
% \affil[4]{Chair of Applied Mechanics,
%   Friedrich-Alexander-Universit\"{a}t Erlangen-N\"{u}rnberg,
%   Egerlandstr.\ 5,
%   91058 Erlangen, Germany.
%   {\texttt{\{denis.davydov,jean-paul.pelteret\}@fau.de}}}
%
% \author[5]{Marc~Fehling}
% \affil[5]{Institute for Advanced Simulation,
%   Forschungszentrum J\"{u}lich GmbH,
%   52425 J\"{u}lich, Germany.
%   {\texttt{m.fehling@fz-juelich.de}}}
%
% \author[6]{Daniel Garcia-Sanchez}
% \affil[6]{Sorbonne Universit\'es, UPMC Univ.\ Paris 06, CNRS-UMR 7588,
%   Institut des NanoSciences de Paris, F-75005, Paris, France
%   {\texttt{daniel.garcia-sanchez@insp.upmc.fr}}}
%
% \author[2]{Graham Harper}
%
\author[3]{Timo~Heister}
 \affil[3]{School of Mathematical and Statistical Sciences,
   Clemson University,
   Clemson, SC, 29634, USA
   {\texttt{heister@clemson.edu}}}

% %
% \author[8]{Luca~Heltai}
% \affil[8]{SISSA,
%   International School for Advanced Studies,
%   Via Bonomea 265,
%   34136, Trieste, Italy.
%   {\texttt{luca.heltai@sissa.it}}}
%
 \author[9]{Martin~Kronbichler}
 \affil[9]{Institute for Computational Mechanics,
   Technical University of Munich,
   Boltzmannstr.~15, 85748 Garching, Germany.
   {\texttt{kronbichler@lnm.mw.tum.de}}}
%
% \author[10]{Ross~Maguire~Kynch}
% \affil[10]{Zienkiewicz Centre for Computational Engineering,
%   College of Engineering, Swansea University,
%   Bay Campus, Fabian Way, Swansea SA1 8EN, Wales, UK.
%   {\texttt{rkynch@gmail.com}}}
%
\author[11]{Matthias~Maier}
\affil[11]{Department of Mathematics,
  Texas A\&M University,
  3368 TAMU,
  College Station, TX 77845, USA.
  {\texttt{maier@math.tamu.edu}}}
%
% \author[4]{Jean-Paul~Pelteret}
%
% \author[*1]{Bruno~Turcksin}
%
% \author[12]{David Wells}
% \affil[12]{Department of Mathematics, University of North Carolina,
%   Chapel Hill, NC 27516, USA.
%   {\texttt{drwells@email.unc.edu}}}

\renewcommand{\labelitemi}{--}


\begin{document}
\maketitle

\footnotetext{%
  $^\ast$ This manuscript has been authored by UT-Battelle, LLC under Contract No.
  DE-AC05-00OR22725 with the U.S. Department of Energy. The United States
  Government retains and the publisher, by accepting the article for
  publication, acknowledges that the United States Government retains a
  non-exclusive, paid-up, irrevocable, worldwide license to publish or reproduce
  the published form of this manuscript, or allow others to do so, for United
  States Government purposes. The Department of Energy will provide public
  access to these results of federally sponsored research in accordance with the
  DOE Public Access Plan (http://energy.gov/downloads/doe-public-access-plan).}

\begin{abstract}
  This paper provides an overview of the new features of the finite element
  library \dealii, version 9.2.
\end{abstract}



%%%%%%%%%%%%%%%%%%%%%%%%%%%%%%%%%%%%%%%%%%%%%%%%%%%%%%%%%%%%%%%%%%%%%%%%%%%%%%%%
%%%%%%%%%%%%%%%%%%%%%%%%%%%%%%%%%%%%%%%%%%%%%%%%%%%%%%%%%%%%%%%%%%%%%%%%%%%%%%%%
%%%%%%%%%%%%%%%%%%%%%%%%%%%%%%%%%%%%%%%%%%%%%%%%%%%%%%%%%%%%%%%%%%%%%%%%%%%%%%%%
\section{Overview}

\dealii version 9.2.0 was released May XYZ, 2020.
This paper provides an
overview of the new features of this release and serves as a citable
reference for the \dealii software library version 9.2. \dealii is an
object-oriented finite element library used around the world in the
development of finite element solvers. It is available for free under the
GNU Lesser General Public License (LGPL). Downloads are available at
\url{https://www.dealii.org/} and \url{https://github.com/dealii/dealii}.

The major changes of this release are:
%
\begin{itemize}
\item xy \todo[inline]{Update once we have the subsections in Section 2;
  provide cross-references to each of these subsections}
%   \item Improved support for automatic differentiation (see
%     Section~\ref{subsec:ad}), 
\end{itemize}
%
The major changes are discussed in detail in Section~\ref{sec:major}. There
are a number of other noteworthy changes in the current \dealii{} release
that we briefly outline in the remainder of this section:
%
\begin{itemize}
\item x \todo[inline]{Wolfgang to write about complex-valued output}
\item y \todo[inline]{Timo to write about problems and fixes for
  computations with more than $2^{32}$ unknowns}
\item z \todo[inline]{What else? Maybe mention the updated step-12?}
\end{itemize}
%
In addition to these changes, the changelog lists more than 200 other
features and bugfixes.




%%%%%%%%%%%%%%%%%%%%%%%%%%%%%%%%%%%%%%%%%%%%%%%%%%%%%%%%%%%%%%%%%%%%%%%%%%%%%%%%
%%%%%%%%%%%%%%%%%%%%%%%%%%%%%%%%%%%%%%%%%%%%%%%%%%%%%%%%%%%%%%%%%%%%%%%%%%%%%%%%
%%%%%%%%%%%%%%%%%%%%%%%%%%%%%%%%%%%%%%%%%%%%%%%%%%%%%%%%%%%%%%%%%%%%%%%%%%%%%%%%
\section{Major changes to the library}
\label{sec:major}

This release of \dealii contains a number of large and significant changes
that will be discussed in this section.

It of course also contains a
vast number of smaller changes and added functionality; the details of these
can be found
\href{https://dealii.org/developer/doxygen/deal.II/changes_between_9_0_1_and_9_1_0.html}{
in the file that lists all changes for this release}, see \cite{changes91}.

%%%%%%%%%%%%%%%%%%%%%%%%%%%%%%%%%%%%%%%%%%%%%%%%%%%%%%%%%%%%%%%%%%%%%%%%%%%%%%%%
\subsection{New and improved tutorial and code gallery programs}
\label{subsec:steps}

Many of the \dealii{} tutorial programs were revised in a variety of
ways as part of this release. A particular example is that we have
converted a number of programs to use range-based for loops (a C++11
feature) for loops over a range of integer indices, given that the
range-based way of writing loops seems to be the idiomatic approach
these days.

In addition, there are a number of new tutorial programs:
\begin{itemize}
\item \texttt{step-47}
\todo[inline]{Zhuoran to write}
\item \texttt{step-50}
\todo[inline]{Timo/Conrad/... to write}
\item \texttt{step-58}
\todo[inline]{Wolfgang to write}  
\item \texttt{step-65}
\todo[inline]{Martin to write}  
\item \texttt{step-67}
\todo[inline]{Martin to write}  
\item \texttt{step-69}
\todo[inline]{Matthias/Ignacio to write}  
\item \texttt{step-70}
  \todo[inline]{Also need to update announce and announce-short if this
    makes it into the release.}
\end{itemize}

\todo[inline]{Do we have new code gallery programs}

%%%%%%%%%%%%%%%%%%%%%%%%%%%%%%%%%%%%%%%%%%%%%%%%%%%%%%%%%%%%%%%%%%%%%%%%%%%%%%%%
\subsection{Support for large, fully distributed meshes}
\label{subsec:pfT}

\todo[inline]{Peter: Write something about p::f::T}



%%%%%%%%%%%%%%%%%%%%%%%%%%%%%%%%%%%%%%%%%%%%%%%%%%%%%%%%%%%%%%%%%%%%%%%%%%%%%%%%
\subsection{Better support for parallel $hp$-adaptive algorithms}
\label{subsec:hp}

\todo[inline]{Marc: Your section}


%%%%%%%%%%%%%%%%%%%%%%%%%%%%%%%%%%%%%%%%%%%%%%%%%%%%%%%%%%%%%%%%%%%%%%%%%%%%%%%%
\subsection{Support for particle-based methods}
\label{subsec:particles}

\todo[inline]{Luca: Your section}


%%%%%%%%%%%%%%%%%%%%%%%%%%%%%%%%%%%%%%%%%%%%%%%%%%%%%%%%%%%%%%%%%%%%%%%%%%%%%%%%
\subsection{Python interfaces}
\label{subsec:python}

\todo[inline]{What's new here? mention step-49 and step-53 versions written in python.}




%%%%%%%%%%%%%%%%%%%%%%%%%%%%%%%%%%%%%%%%%%%%%%%%%%%%%%%%%%%%%%%%%%%%%%%%%%%%%%%%
\subsection{Incompatible changes}

The 9.2 release includes
\href{https://dealii.org/developer/doxygen/deal.II/changes_between_9_1_1_and_9_2_0.html}
     {around 60 incompatible changes}; see \cite{changes92}. The majority of these changes
should not be visible to typical user codes; some remove previously
deprecated classes and functions; and the majority change internal
interfaces that are not usually used in external
applications. However, some are worth mentioning:
\begin{itemize}
\item
\end{itemize}



%%%%%%%%%%%%%%%%%%%%%%%%%%%%%%%%%%%%%%%%%%%%%%%%%%%%%%%%%%%%%%%%%%%%%%%%%%%%%%%%
%%%%%%%%%%%%%%%%%%%%%%%%%%%%%%%%%%%%%%%%%%%%%%%%%%%%%%%%%%%%%%%%%%%%%%%%%%%%%%%%
%%%%%%%%%%%%%%%%%%%%%%%%%%%%%%%%%%%%%%%%%%%%%%%%%%%%%%%%%%%%%%%%%%%%%%%%%%%%%%%%
\section{How to cite \dealii}\label{sec:cite}

In order to justify the work the developers of \dealii put into this
software, we ask that papers using the library reference one of the
\dealii papers. This helps us justify the effort we put into it.

There are various ways to reference \dealii. To acknowledge the use of
the current version of the library, \textbf{please reference the present
document}. For up to date information and a bibtex entry for this document
see:
\begin{center}
 \url{https://www.dealii.org/publications.html}
\end{center}

The original \texttt{\dealii} paper containing an overview of its
architecture is \cite{BangerthHartmannKanschat2007}. If you rely on
specific features of the library, please consider citing any of the
following:
\begin{itemize}
 \item For geometric multigrid: \cite{Kanschat2004,JanssenKanschat2011,ClevengerHeisterKanschatKronbichler2019};
 \item For distributed parallel computing: \cite{BangerthBursteddeHeisterKronbichler11};
 \item For $hp$~adaptivity: \cite{BangerthKayserHerold2007};
  \item For partition-of-unity (PUM) and enrichment methods of the
    finite element space: \cite{Davydov2016};
 \item For matrix-free and fast assembly techniques:
   \cite{KronbichlerKormann2012,KronbichlerKormann2019};
 \item For computations on lower-dimensional manifolds:
   \cite{DeSimoneHeltaiManigrasso2009};
 \item For integration with CAD files and tools:
   \cite{HeltaiMola2015};
 \item For Boundary Elements Computations:
   \cite{GiulianiMolaHeltai-2018-a};
 \item For \texttt{LinearOperator} and \texttt{PackagedOperation} facilities:
   \cite{MaierBardelloniHeltai-2016-a,MaierBardelloniHeltai-2016-b}.
 \item For uses of the \texttt{WorkStream} interface:
   \cite{TKB16};
   \item For uses of the \texttt{ParameterAcceptor} concept, the
     \texttt{MeshWorker::ScratchData} base class, and the
     \texttt{ParsedConvergenceTable} class: \cite{SartoriGiulianiBardelloni-2018-a}.
\end{itemize}

\dealii can interface with many other libraries:
\begin{multicols}{3}
\begin{itemize}
\item ADOL-C \cite{Griewank1996a,adol-c}
\item ARPACK \cite{arpack}
\item Assimp \cite{assimp}
\item BLAS and LAPACK \cite{lapack}
\item cuSOLVER \cite{cusolver}
\item cuSPARSE \cite{cusparse}
\item Gmsh \cite{geuzaine2009gmsh}
\item GSL \cite{gsl2016}
\item Ginkgo \cite{ginkgo-web-page}
\item HDF5 \cite{hdf5}
\item METIS \cite{karypis1998fast}
\item MUMPS \cite{ADE00,MUMPS:1,MUMPS:2,mumps-web-page}
\item muparser \cite{muparser-web-page}
\item nanoflann \cite{nanoflann}
\item NetCDF \cite{rew1990netcdf}
\item OpenCASCADE \cite{opencascade-web-page}
\item p4est \cite{p4est}
\item PETSc \cite{petsc-user-ref,petsc-web-page}
\item ROL \cite{ridzal2014rapid}
\item ScaLAPACK \cite{slug}
\item SLEPc \cite{Hernandez:2005:SSF}
\item SUNDIALS \cite{sundials}
\item SymEngine \cite{symengine-web-page}
\item TBB \cite{Rei07}
\item Trilinos \cite{trilinos,trilinos-web-page}
\item UMFPACK \cite{umfpack}
\end{itemize}
\end{multicols}
Please consider citing the appropriate references if you use interfaces to these
libraries.

The two previous releases of \dealii can be cited as
\cite{dealII90,dealII91}.


\section{Acknowledgments}

\dealii is a world-wide project with dozens of contributors around the
globe. Other than the authors of this paper, the following people
contributed code to this release:\\
% updated 5/11/2020 MM
\todo[inline]{TODO: remove authors of paper}
Pasquale Africa,
Ashna Aggarwal,
Giovanni Alzetta,
Mathias Anselmann,
Daniel Arndt,
Wolfgang Bangerth,
Kirana Bergstrom,
Manaswinee Bezbaruah,
Bruno Blais,
Benjamin Brands,
Yong-Yong Cai,
Fabian Castelli,
Thomas C. Clevenger,
Katherine Cosburn,
Denis Davydov,
Elias Dejene,
Stefano Dominici,
Brett Dong,
Luel Emishaw,
Marc Fehling,
Niklas Fehn,
Rebecca Fildes,
Menno Fraters,
Andres Galindo,
Daniel Garcia-Sanchez,
Rene Gassmoeller,
Melanie Gerault,
Nicola Giuliani,
Brandon Gleeson,
Anne Glerum,
Krishnakumar Gopalakrishnan,
Alexander Grayver,
Graham Harper,
Mohammed Hassan,
Nicole Hayes,
Bang He,
Johannes Heinz,
Timo Heister,
Luca Heltai,
Jiuhua Hu,
Lise-Marie Imbert-Gerard,
Manu Jayadharan,
Daniel Jodlbauer,
Marie Kajan,
Guido Kanschat,
Alexander Knieps,
Martin Kronbichler,
Paras Kumar,
Konstantin Ladutenko,
Charu Lata,
Adam Lee,
Wenyu Lei,
Matthias Maier,
Katrin Mang,
Mae Markowski,
Franco Milicchio,
Adriana Morales Miranda,
Peter Munch,
Bob Myhill,
Emily Novak,
Omotayo Omosebi,
Alexey Ozeritskiy,
Jean-Paul Pelteret,
Rebecca Pereira,
Geneva Porter,
Laura Prieto Saavedra,
Reza Rastak,
Roland Richter,
Jonathan Robey,
Irabiel Romero,
Matthew Russell,
Tonatiuh Sanchez-Vizuet,
Natasha S. Sharma,
Doug Shi-Dong,
Konrad Simon,
Stephanie Sparks,
Sebastian Stark,
Simon Sticko,
Jan Philipp Thiele,
Jihuan Tian,
Ignacio Tomas,
Sara Tro,
Bruno Turcksin,
Ferdinand Vanmaele,
Zhuoran Wang,
David Wells,
Michal Wichrowski,
Julius Witte,
Winnifried Wollner,
Ming Yang,
Mario Zepeda Aguilar,
Wenjuan Zhang,
Victor Zheng.

Their contributions are much appreciated!


\bigskip

\dealii and its developers are financially supported through a
variety of funding sources:

D.~Arndt and M.~Kronbichler were partially supported by the German
Research Foundation (DFG) under the project ``High-order discontinuous
Galerkin for the exa-scale'' (\mbox{ExaDG}) within the priority program ``Software
for Exascale Computing'' (SPPEXA).

W.~Bangerth, T.~C.~Clevenger, and T.~Heister were partially
supported by the National Science Foundation under award OAC-1835673
as part of the Cyberinfrastructure for Sustained Scientific Innovation (CSSI)
program  and by the Computational Infrastructure
in Geodynamics initiative (CIG), through the National Science
Foundation under Award No.~EAR-1550901 and The
University of California -- Davis.

W.~Bangerth and T.~Heister were also partially supported by award DMS-1821210.

D.~Davydov was supported by the German Research Foundation (DFG), grant DA
1664/2-1 and the Bayerisches Kompetenznetzwerk
f\"ur Technisch-Wissenschaftliches Hoch- und H\"ochstleistungsrechnen
(KONWIHR).

T.~Heister was also partially supported by NSF Award DMS-1522191, and
by Technical Data Analysis, Inc. through US Navy SBIR N16A-T003.

M.~Kronbichler was also supported by the Bayerisches Kompetenznetzwerk
f\"ur Technisch-Wissen\-schaft\-li\-ches Hoch- und H\"ochstleistungsrechnen
(KONWIHR) in the context of the project
``Performance tuning of high-order discontinuous Galerkin solvers for
SuperMUC-NG''.

R.~M.~Kynch was supported by the Engineering and Physical Science Research
Council (EPSRC) UK through grant EP/K023950/1 while working at Swansea
University, UK 2013-2015 where the majority of his contribution was
undertaken.

M.~Maier was partially supported by ARO MURI Award No. W911NF-14-0247 and
NSF Award DMS-1912847.

B.~Turcksin: Research sponsored by the Laboratory Directed Research and
Development Program of Oak Ridge National Laboratory, managed by UT-Battelle,
LLC, for the U. S. Department of Energy.

D.~Wells was supported by the National Science Foundation (NSF) through Grant
DMS-1344962.

The Interdisciplinary Center for Scientific Computing (IWR) at Heidelberg
University has provided hosting services for the \dealii web page.


\bibliography{paper}{}
\bibliographystyle{abbrv}

\end{document}

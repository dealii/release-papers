\documentclass{ansarticle-preprint}
%\usepackage{ucs}
\usepackage[utf8]{inputenc}
\usepackage{amsmath}
%\usepackage{cite}
\usepackage{anslistings}
\usepackage{multicol}
\usepackage{pdfsync}

\usepackage{pgfplots}
\usepackage{pgfplotstable}

\usepackage{fontenc}
\usepackage{graphicx}
\usepackage{xspace}

%\renewcommand{\baselinestretch}{2.0}

\usepackage[normalem]{ulem}

\pgfplotsset{compat=1.9}

\newcommand{\specialword}[1]{\texttt{#1}}
\newcommand{\dealii}{{\specialword{deal.II}}\xspace}
\newcommand{\pfrst}{{\specialword{p4est}}\xspace}
\newcommand{\trilinos}{{\specialword{Trilinos}}\xspace}
\newcommand{\aspect}{\specialword{Aspect}\xspace}
\newcommand{\petsc}{\specialword{PETSc}\xspace}
\newcommand{\cmake}{{\specialword{CMake}}\xspace}
\newcommand{\autoconf}{{\specialword{autoconf}}\xspace}

%
% Author list -- please add yourself in both places below (in
%                alphabetical order) if you think that your
%                contributions to the last release warrant this
%

\hypersetup{
  pdfauthor={
     Daniel Arndt,
     Wolfgang Bangerth,
%     Thomas C. Clevenger,
%     Denis Davydov,
%     Marc Fehling,
%     Daniel Garcia-Sanchez,
%     Graham Harper,
    Timo Heister,
%     Luca Heltai,
    Martin Kronbichler,
    Peter Munch,
%     Ross Maguire Kynch,
%     Matthias Maier,
%     Jean-Paul Pelteret,
%     Bruno Turcksin,
%     David Wells
  },
  pdftitle={The deal.II Library, Version 9.2, 2020},
}

\title{The \dealii Library, Version 9.2}

 \author[*1]{Daniel Arndt}
 \affil[1]{Computational Engineering and Energy Sciences Group,
   Computional Sciences and Engineering Division,
   Oak Ridge National Laboratory, 1 Bethel Valley Rd.,
   TN 37831, USA.
   \texttt{arndtd@ornl.gov}}

 \author[2]{Wolfgang~Bangerth}
 \affil[2]{Department of Mathematics, Colorado State University, Fort
   Collins, CO 80523-1874, USA.
   \texttt{bangerth@colostate.edu}}

% \author[3]{Thomas~C.~Clevenger}
% \affil[3]{School of Mathematical and Statistical Sciences,
%   Clemson University,
%   Clemson, SC, 29634, USA
%   {\texttt{\{tcleven, heister\}@clemson.edu}}}
%
% \author[4]{Denis~Davydov}
% \affil[4]{Chair of Applied Mechanics,
%   Friedrich-Alexander-Universit\"{a}t Erlangen-N\"{u}rnberg,
%   Egerlandstr.\ 5,
%   91058 Erlangen, Germany.
%   {\texttt{\{denis.davydov,jean-paul.pelteret\}@fau.de}}}
%
% \author[5]{Marc~Fehling}
% \affil[5]{Institute for Advanced Simulation,
%   Forschungszentrum J\"{u}lich GmbH,
%   52425 J\"{u}lich, Germany.
%   {\texttt{m.fehling@fz-juelich.de}}}
%
% \author[6]{Daniel Garcia-Sanchez}
% \affil[6]{Sorbonne Universit\'es, UPMC Univ.\ Paris 06, CNRS-UMR 7588,
%   Institut des NanoSciences de Paris, F-75005, Paris, France
%   {\texttt{daniel.garcia-sanchez@insp.upmc.fr}}}
%
% \author[2]{Graham Harper}
%
\author[3]{Timo~Heister}
 \affil[3]{School of Mathematical and Statistical Sciences,
   Clemson University,
   Clemson, SC, 29634, USA
   {\texttt{heister@clemson.edu}}}

% %
% \author[8]{Luca~Heltai}
% \affil[8]{SISSA,
%   International School for Advanced Studies,
%   Via Bonomea 265,
%   34136, Trieste, Italy.
%   {\texttt{luca.heltai@sissa.it}}}
%
 \author[9]{Martin~Kronbichler}
 \affil[9]{Institute for Computational Mechanics,
   Technical University of Munich,
   Boltzmannstr.~15, 85748 Garching, Germany.
   {\texttt{kronbichler/munch@lnm.mw.tum.de}}}
%
% \author[10]{Ross~Maguire~Kynch}
% \affil[10]{Zienkiewicz Centre for Computational Engineering,
%   College of Engineering, Swansea University,
%   Bay Campus, Fabian Way, Swansea SA1 8EN, Wales, UK.
%   {\texttt{rkynch@gmail.com}}}
%
\author[11]{Matthias~Maier}
\affil[11]{Department of Mathematics,
  Texas A\&M University,
  3368 TAMU,
  College Station, TX 77845, USA.
  {\texttt{maier@math.tamu.edu}}}
  
  \author[9,13]{Peter Munch}
  

 \affil[13]{Institute of Materials Research, Materials Mechanics, 
 Helmholtz-Zentrum Geesthacht, 
 Max-Planck-Str. 1, 21502 Geesthacht, Germany.
   {\texttt{peter.muench@hzg.de}}}  
  
%
% \author[4]{Jean-Paul~Pelteret}
%
% \author[*1]{Bruno~Turcksin}
%
% \author[12]{David Wells}
% \affil[12]{Department of Mathematics, University of North Carolina,
%   Chapel Hill, NC 27516, USA.
%   {\texttt{drwells@email.unc.edu}}}

\renewcommand{\labelitemi}{--}


\begin{document}
\maketitle

\footnotetext{%
  $^\ast$ This manuscript has been authored by UT-Battelle, LLC under Contract No.
  DE-AC05-00OR22725 with the U.S. Department of Energy. The United States
  Government retains and the publisher, by accepting the article for
  publication, acknowledges that the United States Government retains a
  non-exclusive, paid-up, irrevocable, worldwide license to publish or reproduce
  the published form of this manuscript, or allow others to do so, for United
  States Government purposes. The Department of Energy will provide public
  access to these results of federally sponsored research in accordance with the
  DOE Public Access Plan (http://energy.gov/downloads/doe-public-access-plan).}

\begin{abstract}
  This paper provides an overview of the new features of the finite element
  library \dealii, version 9.2.
\end{abstract}



%%%%%%%%%%%%%%%%%%%%%%%%%%%%%%%%%%%%%%%%%%%%%%%%%%%%%%%%%%%%%%%%%%%%%%%%%%%%%%%%
%%%%%%%%%%%%%%%%%%%%%%%%%%%%%%%%%%%%%%%%%%%%%%%%%%%%%%%%%%%%%%%%%%%%%%%%%%%%%%%%
%%%%%%%%%%%%%%%%%%%%%%%%%%%%%%%%%%%%%%%%%%%%%%%%%%%%%%%%%%%%%%%%%%%%%%%%%%%%%%%%
\section{Overview}

\dealii version 9.2.0 was released May XYZ, 2020.
This paper provides an
overview of the new features of this release and serves as a citable
reference for the \dealii software library version 9.2. \dealii is an
object-oriented finite element library used around the world in the
development of finite element solvers. It is available for free under the
GNU Lesser General Public License (LGPL). Downloads are available at
\url{https://www.dealii.org/} and \url{https://github.com/dealii/dealii}.

The major changes of this release are:
%
\begin{itemize}
\item
%   \item Improved support for automatic differentiation (see
%     Section~\ref{subsec:ad}),
%   \item Dedicated support for symbolic algebra (see
%     Section~\ref{subsec:sd}),
%   \item Full support for $hp$~adaptivity in parallel computations (see
%     Section~\ref{subsec:hp}),
%   \item An interface to the HDF5 file format and libraries (see
%     Section~\ref{subsec:hdf5}),
%   \item Significantly extended GPU support (see Section~\ref{subsec:gpu}),
%   \item Parallel geometric multigrid (GMG) improvements (see
%     \cite{ClevengerHeisterKanschatKronbichler2019} and
%     Section~\ref{subsec:gmg}),
%   \item Four new tutorial programs (step-61, step-62, step-63, step-64),
%     as well as one new code gallery program (see
%     Section~\ref{subsec:steps}).
\end{itemize}
%
The major changes are discussed in detail in Section~\ref{sec:major}. There
are a number of other noteworthy changes in the current \dealii{} release
that we briefly outline in the remainder of this section:
%
\begin{itemize}
\item
% \item
%   The release contains a number of performance improvements and bug fixes for
%   the matrix-free framework. One notable improvement is the support for
%   renumbering of degrees of freedom within the cells for discontinuous
%   elements, avoiding some reshuffling operations across the SIMD lanes
%   with vectorization over several cells and faces, which is especially
%   useful on processors with AVX-512 vectorization (8 doubles), speeding up
%   operations by up to 10\%. Secondly, the strategy for the most efficient
%   tensor product evaluators according to the performance analysis of
%   \cite{KronbichlerKormann2019} in the context of more quadrature points than
%   shape functions has been revised for better performance.
%
% \item A new class \texttt{ParsedConvergenceTable} has been introduced
%   that greatly simplifies the construction of convergence tables,
%   reading the options for the generation of the table from a parameter
%   file, and providing methods that, combined with a parameter file,
%   allow one to generate convergence tables using one-liners in user
%   codes.
%
% \item
%   The \texttt{FE\_BernardiRaugel} class implements the non-standard
%   Bernardi-Raugel (BR) element that can be used to construct a stable
%   velocity-pressure pair for the Stokes equation \cite{BR85}. The BR
%   element is an enriched version of the $Q_1^d$ element with added bubble
%   functions on each edge (in 2d) or face (in 3d). It addresses the fact
%   that the $Q_1^d\times Q_0$ combination is not inf-sup stable (requiring a
%   larger velocity space), and that the $Q_2^d\times Q_0$ combination is
%   stable but converges with only first-order at the cost of the large
%   number of velocity unknowns. The BR space is thus intermediate between the
%   $Q_1^d$ and $Q_2^d$ spaces.
%
%   The element is currently only implemented for parallelogram meshes due to
%   difficulties associated with the mapping of shape functions: The shape
%   functions of the $Q_1^d$ part of the element need to be mapped as
%   scalars, as is common for the vector components of the $Q_1^d$ element;
%   on the other hand, the vector-valued edge bubble functions need to be
%   mapped using the Piola transform as is common for the Raviart-Thomas
%   element. \dealii{} does not currently have the ability to use different
%   mappings for individual shape functions, though this functionality is
%   planned for the next release.
%
% \item
%   The \texttt{FE\_NedelecSZ} class is a new implementation of the
%   N{\'e}d{\'e}lec element on quadrilaterals and hexahedra. It is based on
%   the work of Zaglmayr \cite{Zag06} and overcomes the sign conflict issues
%   present in traditional N{\'e}d{\'e}lec elements that arise from the edge
%   and face parameterizations used in the basis functions. Therefore, this
%   element should provide consistent results for general quadrilateral and
%   hexahedral elements for which the relative orientations of edges and
%   faces (as seen from all adjacent cells) are often difficult to establish.
%   The \texttt{FE\_NedelecSZ} element addresses the sign conflict problem by
%   assigning a globally defined orientation to local edges and faces. A
%   detailed overview of the implementation of the \texttt{FE\_NedelecSZ}
%   element in \dealii{} can be found in \cite{Kynch2017}.
%
% \item All of the elementary geometrical objects of the library (namely
%   \texttt{Point<dim>}, \texttt{Segment<dim>}, and
%   \texttt{BoundingBox<dim>}) have been augmented with the traits
%   needed to comply with \texttt{boost::geometry} concepts. A new
%   interface to \texttt{boost::geometry::index::rtree} has been added
%   that simplifies the construction of spatial indices based on points,
%   bounding boxes, or segments.
\end{itemize}
%
In addition to these changes, the changelog lists more than 200 other
features and bugfixes.




%%%%%%%%%%%%%%%%%%%%%%%%%%%%%%%%%%%%%%%%%%%%%%%%%%%%%%%%%%%%%%%%%%%%%%%%%%%%%%%%
%%%%%%%%%%%%%%%%%%%%%%%%%%%%%%%%%%%%%%%%%%%%%%%%%%%%%%%%%%%%%%%%%%%%%%%%%%%%%%%%
%%%%%%%%%%%%%%%%%%%%%%%%%%%%%%%%%%%%%%%%%%%%%%%%%%%%%%%%%%%%%%%%%%%%%%%%%%%%%%%%
\section{Major changes to the library}
\label{sec:major}

This release of \dealii contains a number of large and significant changes
that will be discussed in this section.

It of course also contains a
vast number of smaller changes and added functionality; the details of these
can be found
\href{https://dealii.org/developer/doxygen/deal.II/changes_between_9_0_1_and_9_1_0.html}{
in the file that lists all changes for this release}, see \cite{changes91}.

%%%%%%%%%%%%%%%%%%%%%%%%%%%%%%%%%%%%%%%%%%%%%%%%%%%%%%%%%%%%%%%%%%%%%%%%%%%%%%%%
\subsection{bla1}
\label{subsec:bla1}


%%%%%%%%%%%%%%%%%%%%%%%%%%%%%%%%%%%%%%%%%%%%%%%%%%%%%%%%%%%%%%%%%%%%%%%%%%%%%%%%
\subsection{Improved large-scale performance}
\label{subsec:performance}

Large-scale simulations with 304,128 cores have revealed bottlenecks in release 
9.1 during initialization due to the usage of expensive collective operations 
like \texttt{MPI\_Allgather()} and \texttt{MPI\_\allowbreak Alltoall()}. These 
operations are used to retrieve or even store information about all processes. 
For example, all processes have stored in  an array the number of degrees of 
freedom each process owns. This information is in particular needed to set up 
the  \texttt{Utilities::MPI::Partitioner} class, which contains the 
point-to-point communication pattern for vector ghost-value updates and 
compressions. In release 9.2, we have removed such arrays and have replaced 
the \texttt{MPI\_Allgather}/\allowbreak\texttt{MPI\_\allowbreak Alltoall} 
function calls by consensus algorithms~\cite{hoefler2010scalable}, which can be 
found in the namespace \texttt{Utilities::\allowbreak MPI::\allowbreak ConsensusAlgorithms}: now, only the locally relevant information is computed 
(and recomputed) when needed, using these algorithms. 

Consensus algorithms are algorithms dedicated to efficient dynamic-sparse 
communication patterns. In this context, the term ``dynamic-sparse'' means 
that by the time this function is called, the other processes do not know 
yet that they have to answer requests and
each process only has to communicate with a small subset of processes of the 
MPI communicator. We provide two flavors of the consensus algorithm: the two-step 
approach \texttt{ConsensusAlgorithms::PEX} and the \texttt{ConsensusAlgorithms::NBX}, 
which uses only point-to-point communications and a single \texttt{MPI\_IBarrier()}. 
The class \texttt{ConsensusAlgorithms::Selector} selects one of the two previous 
algorithms, depending on the number of processes.

Due to the excellent scalability of the consensus algorithms, users are encouraged 
to use them for their own dynamic-sparse problems by providing a list of target 
processes and pack/unpack routines either by implementing the interface 
\texttt{ConsensusAlgorithms::Process} or by providing \texttt{std::function} 
objects to \texttt{ConsensusAlgorithms::AnonymousProcess}.


The \texttt{ConsensusAlgorithms} are used by now internally in many places. These 
places are appropriate starting points for users for their own application of the 
\texttt{ConsensusAlgorithms} infrastructure.
For example, to set up the partitioners, the new function 
\texttt{compute\_index\_owner() } is used: given an index set containing the 
locally owned indices and an index set containing the ghost indices, it returns 
the owner of the ghost indices. Consensus algorithms are used now also in the 
functions \texttt{compute\_point\_to\_point\_communication\_pattern()} and 
\texttt{compute\_\allowbreak n\_\allowbreak point\_\allowbreak to\_\allowbreak point\_\allowbreak communications()}. 
Furthermore, it is utilized in the class \texttt{NoncontiguousPartitioner} to 
efficiently permute distributed solution vectors globally in an arbitrary order, e.g., 
to interface with external libraries that prescribe a certain partitioning and 
padding of the data.

By replacing the collective communications during setup and removing the arrays 
that contain information for each process (enabled by the application of consensus 
algorithms and other modifications---a full list of modifications leading to this 
improvement can be found online), we were able to significantly improve the setup 
time for large-scale simulations and to solve a Poisson problem with multigrid 
with 12T unknowns. 
Figure~\ref{} compares the timings of a simulation (incl. setup) with the 
previous release 9.1 and with the current release 9.2. 
{\color{red}TODO[Peter/Martin] description of the results}

The new code has been also applied to solve problems with adaptively refined 
meshes with more than 4B unknowns. {\color{red}TODO[Timo]}


%%%%%%%%%%%%%%%%%%%%%%%%%%%%%%%%%%%%%%%%%%%%%%%%%%%%%%%%%%%%%%%%%%%%%%%%%%%%%%%%
\subsection{A new fully distributed triangulation class}
\label{subsec:pft}

By release 9.1, \texttt{deal.II} had three types of triangulation classes: the 
serial triangulation class \texttt{Triangulation} as well as the parallel 
triangulation classes \texttt{parallel::shared::Triangulation} and \texttt{parallel::distributed::Triangulation}. The latter builds around a serial 
triangulation and uses \texttt{p4est} as an oracle during adaptive mesh refinement. 
All these triangulation classes have in common that the coarse grid is shared by 
all processes and the actual mesh used for computations is constructed by repeated 
local and/or global refinement, which adapts nicely to curved boundaries described 
by the \texttt{Manifold} class. However, this way to construct a computational 
mesh has its limitations in industrial applications where, often, the mesh comes 
from an external CAD program in the form of a file that already contains millions 
or tens of millions of cells with a similar number of vertices. In such a case, 
refining a mesh is not practical, since it would increase the computational effort 
and new vertices would not be placed on curved boundaries. A problem that arises 
for such large grids in the context how meshes have been treated in \texttt{deal.II} 
until now is that the coarse grid, i.e., potentially the whole mesh is shared by 
all processes. It might be a major difficulty in MPI-only parallelized applications 
on modern multi-core processors, since these applications might already run out of 
main memory during reading the mesh. Not even increasing the number of processes 
might help in this situation. 

The new class \texttt{parallel::fullydistributed::Triangulation} targets this issue 
by distributing also the coarse grid, which is the reason for the name of the 
chosen namespace: it distributes the coarse grid as well as the refinement levels. Such 
a triangulation can be created by providing a \texttt{TriangulationDescription::Description} struct to each process, containing 
1) the relevant data to construct the local part of the coarse grid, 2) the 
translation of the local coarse-cell IDs to globally unique IDs, 3) the hierarchy 
of mesh refinements, and 4) the owner of the cells on the active mesh level as well 
as on the multigrid levels. Once the triangulation is set up with this struct, no 
changes to the mesh are allowed at the moment.
 
The \texttt{TriangulationDescription::Description} struct can be filled manually or 
by the utility functions from the \texttt{TriangulationDescription::Utilities} 
namespace. The function \texttt{create\_\allowbreak description\_\allowbreak 
from\_ \allowbreak triangulation()} can convert a base triangulation (partitioned 
serial \texttt{Tri\-angulation} and \texttt{parallel::distributed::Triangulation}) 
to such a struct. The advantage of this approach is that all known utility 
functions from the namespaces \texttt{GridIn} and \texttt{GridTools} can be used 
on the base triangulations before converting them to the structs. Since this 
function suffers from the same main memory problems as described above, we also 
provide the function \texttt{create\_description\_from\_triangulation\_in\_groups()}, 
which creates the structs only on the master process in a process group. These 
structs are filled one by one and are sent to the relevant processes once they are 
ready. A sensible process group size might contain all processes of one compute node.


The new (fully) distributed triangulation class works---in contrast to  
\texttt{parallel::distributed::\allowbreak Tri\-an\-gu\-la\-tion}---not only for 2D- and 3D- but also for 
1D-problems. It can be used in the context of geometric multigrid methods and 
supports periodic boundary conditions. Furthermore, hanging nodes are supported.

We intend to extend the usability of the new triangulation class in regard of 
different aspects, e.g., I/O. In addition, we would like to enable adaptive mesh 
refinement, a feature of the other triangulation classes, which is very much 
appreciated by many users. For repartitioning, we plan to use an oracle approach 
known from \texttt{parallel::distributed::Triangulation}. Here, we would like to 
rely on a user-provided partitioner, which might also be a graph partitioner.


%%%%%%%%%%%%%%%%%%%%%%%%%%%%%%%%%%%%%%%%%%%%%%%%%%%%%%%%%%%%%%%%%%%%%%%%%%%%%%%%
\subsection{Advances of the SIMD capabilities and the matrix-free infrastructure}
\label{subsec:mf}

%\begin{itemize}
%\item ECL
%\item VectorizedArrayType
%\end{itemize}

The class \texttt{VectorizedArray<Number>} is a key ingredient for the high 
node-level performance of the matrix-free algorithms in deal.II. It is a wrapper 
class around $n$ vector entries of type \texttt{Number} and delegates relevant 
function calls to appropriate Intrinsics instructions. Up to release 9.1, the 
vector length $n$ has been set at compile time of the library to the highest 
possible value supported by the given processor architecture.

The class \texttt{VectorizedArray} has been made more user-friendly by making 
it compatible with the STL algorithms found in the header \texttt{<algorithm>}. 
Now, it has following features:
\begin{itemize}
\item \texttt{VectorizedArray::size()} returns the vector length. This function 
replaces the public static attribute \texttt{VectorizedArray::n\_array\_elements}, 
which has been deprecated.
\item \texttt{VectorizedArray::value\_type} contains the underlying number type of 
the array.
\item \texttt{VectorizedArray} has an output operator 
\texttt{std::ostream\& operator<<(\&out, \&p)}.
\item \texttt{VectorizedArray::begin()} and \texttt{VectorizedArray::end()} allow 
range-based iteration over all vector entries.
\end{itemize}
Furthermore, the \texttt{VectorizedArray} class supports the following (tested) 
algorithms: \texttt{std::\allowbreak ad\-vance()}, \texttt{std::distance()}, and \texttt{std::max\_element()}.

It has been also extended with the second optional template argument 
\texttt{VectorizedArray<Number, size>} with \texttt{size} being related to the 
vector length, i.e., the number of lanes to be used and the instruction set to be 
used. By default, the number is set to the highest value supported by the given 
hardware, i.e., \texttt{VectorizedArray<double>} is translated on Skylake-based 
processors to \texttt{VectorizedArray<double, 8>}. A full list of supported 
vector lengths are presented in Table~\ref{tab:vectorizedarray}.

All matrix-free related classes (like \texttt{MatrixFree} and \texttt{FEEvaluation}) 
have been templated with the floating-point number type \texttt{Number} (e.g. \texttt{double} or \texttt{float}); the computations were performed implicitly 
on \texttt{VectorizedArray<Number>} structs with the highest 
instruction-set-architecture extension, with each lane responsible for a separate 
cell (vectoriziation over elements). In release 9.2, all matrix-free classes 
have been extended with a new optional template argument specifying the 
\texttt{VectorizedArrayType}. This allows users to select the vector length/ISA and, 
as a consequence, the number of cells to be processed at once. 
\begin{table}
\caption{Supported vector lengths of the class \texttt{VectorizedArray} and 
the corresponding instruction-set-architecture extensions. }\label{tab:vectorizedarray}
\centering
\begin{tabular}{ccc}
\toprule
\textbf{double} & \textbf{float} & \textbf{ISA}\\
\midrule
VectorizedArray<double, 1> & VectorizedArray<float, 1> & (auto-vectorization) \\
VectorizedArray<double, 2> & VectorizedArray<float, 4> & SSE2 \\ 
VectorizedArray<double, 4> & VectorizedArray<float, 8> & AVX/AVX2 \\ 
VectorizedArray<double, 8> & VectorizedArray<float, 16> & AVX-512 \\ 
\bottomrule
\end{tabular}

\caption{Comparison of relevant SIMD-related classes in deal.II and C++20.}\label{tab:simd}
\centering
\begin{tabular}{cc}
\toprule
\textbf{VectorizedArray (deal.II)} & \textbf{std::simd (C++20)} \\
\midrule
VectorizedArray<Number> & std::experimental::native\_simd<Number> \\
VectorizedArray<Number, size> & std::experimental::fixed\_size\_simd<Number, size> \\ \bottomrule
\end{tabular}
\end{table}

In standard (2D/3D) matrix-free applications with moderate polynomial degrees, 
we found that there is no reason to modify the default vector length of 
\texttt{VectorizedArray}, since it reaches the highest possible computational 
throughput despite of increased memory footprint. However, in the deal.II-based 
library \texttt{hyper.deal}~\cite{munch2020hyperdeal}, where the same matrix-free 
infrastructure was used for solving the 6D Vlasov--Poisson equation with high-order 
discontinuous Galerkin methods (with more than 1024 degrees of freedom per cell), the 
benefit of decreasing the number of cells processed by a single SIMD instruction was 
shown. That library works with \texttt{MatrixFree} objects of different SIMD-vector 
length in the same application, which would not have been possible before this 
release.

In a next step, we intend to support that users could set 
\texttt{VectorizedArrayType} to \texttt{Number}, which would lead to the usage not 
of \texttt{VectorizedArray<Number, 1>} but of a specialized code-path exploiting 
vectorization within an element~\cite{KronbichlerKormann2019}.

A side effect of introducing the new template argument \texttt{VectorizedArrayType} 
in the \texttt{MatrixFree} classes is that any data structures 
\texttt{VectorizedArrayType} can be processed if they support required 
functionalities like \texttt{size()} or \texttt{value\_type}. In this context, we 
would like to highlight that the new \texttt{C++20} feature \texttt{std::simd} 
can be processed by the matrix-free infrastructure with minor internal 
adjustment as an open pull request shows 
(see \url{https://github.com/dealii/dealii/pull/9994}).  
Table~\ref{tab:simd} gives a comparison of the deal.II-specific SIMD classes and 
the equivalent C++20 classes. We welcome the standardization of the SIMD 
parallelization paradigm in C++ and intend to replace step by step our own 
wrapper class, which has been continuously developed over the last decade. We 
would like to emphasize that the work invested in this class was not in vain, 
since many performance-relevant utility functions implemented with \texttt{VectorizedArray} in mind (e.g., \texttt{vectorized\_load\_and\_transpose} 
and \texttt{vectorized\_transpose\_and\_store}) will be still used, since they 
have not become part of the standard.

Further additions to the \texttt{MatrixFree} infrastructure consist of:
\begin{itemize}
\item a new variant of \texttt{MatrixFree::cell\_loop()}: It takes two
\texttt{std::function} objects with ranges on the locally owned degrees of freedom, one
with work to be scheduled before the cell operation first touches some
unknowns and another with work to be executed after the cell operation last 
touches them. The goal of
these functions is to bring vector operations close to the time when the
vector entries are accessed by the cell operation, which increases the cache
hit rate of modern processors by improved temporal locality. 
\item a new form of loop \texttt{MatrixFree::loop\_cell\_centric()}: This 
kind of loop can be used in the context of discontinuous Galerkin methods, 
where both cell and face integrals have to be evaluated. While in the case of 
the traditional \texttt{loop}, cell and face integrals have been performed 
independently, the new loop performs all cell and face integrals of a cell in 
one go. This includes that each face integral has to be evaluated twice, but 
entries have to be written into the solution vector only once with improved 
data locality. Previous publications based on \texttt{deal.II} have shown the 
relevance of the latter aspect for reaching higher performance. 
\end{itemize}


%%%%%%%%%%%%%%%%%%%%%%%%%%%%%%%%%%%%%%%%%%%%%%%%%%%%%%%%%%%%%%%%%%%%%%%%%%%%%%%%
\subsection{Advances in GPU support}
\label{subsec:gpu}

\begin{itemize}
\item overlapping of computation and communication in the case of CUDA-aware MPI
\end{itemize}


%%%%%%%%%%%%%%%%%%%%%%%%%%%%%%%%%%%%%%%%%%%%%%%%%%%%%%%%%%%%%%%%%%%%%%%%%%%%%%%%
\subsection{New and improved tutorial and code gallery programs}
\label{subsec:steps}

Many of the \dealii{} tutorial programs were substantially revised as
part of this release. In particular, we have converted many places
that now allow for simpler code through the use of C++11 features such
as range-based for loops and lambda functions.

In addition, there are seven new tutorial programs:
\begin{itemize}
\item \texttt{step-47}
\item \texttt{step-50}
\item \texttt{step-58}
\item \texttt{step-65} presents \texttt{TransfiniteInterpolationManifold}, a
manifold class that can propagate curved boundary information into the
interior of a computational domain, and \texttt{MappingQCache} for fast operations for
expensive manifolds.
\item \texttt{step-67} presents an explicit time integrator for the
compressible Euler equations discretized with a high-order discontinuous
Galerkin scheme using the matrix-free infrastructure. Besides the use of
matrix-free evaluators for systems of equations and over-integration, it also
presents \texttt{MatrixFreeOperators::CellwiseInverseMassMatrix}, a fast implementation
of the action of the inverse mass matrix in the DG setting using tensor
products. Furthermore, this tutorial demonstrates i.a. the usage of the new 
pre and post operations which can be passed to \texttt{cell\_loop()} 
(see also Subsection~\ref{subsec:mf}) and discusses performance-related aspects.
\item \texttt{step-69}
\item \texttt{step-70}
\end{itemize}


%%%%%%%%%%%%%%%%%%%%%%%%%%%%%%%%%%%%%%%%%%%%%%%%%%%%%%%%%%%%%%%%%%%%%%%%%%%%%%%%
\subsection{Incompatible changes}

The 9.2 release includes
\href{https://dealii.org/developer/doxygen/deal.II/changes_between_9_1_1_and_9_2_0.html}
     {around 15 incompatible changes}; see \cite{changes92}. The majority of these changes
should not be visible to typical user codes; some remove previously
deprecated classes and functions; and the majority change internal
interfaces that are not usually used in external
applications. However, some are worth mentioning:
\begin{itemize}
\item The functions:
\begin{itemize}
\item \texttt{DoFHandler::loccaly\_owned\_dofs\_per\_processor()}
\item \texttt{DoFHandler::loccaly\_owned\_mg\_dofs\_per\_processor()}
\end{itemize} have been deprecated. As discussed in Subsection~\ref{subsec:performance}, deal.II does not store information for all processes on all processes processes, but only the local information or the locally-relevant information. Users are asked to construct the global information on their own, e.g. by calling \texttt{Utilities::MPI::Allgather(locally\_owned\_info(), comm)}.
\item 

% \item The \texttt{VectorView} class was removed. We recommend either copying the
%       vector subset into a \texttt{Vector} or using a \texttt{BlockVector}.
% \item The function \texttt{Subscriptor::subscribe()}, used through the
%   \texttt{SmartPointer} class, now requires a pointer to a
%       \texttt{std::atomic<bool>} that tracks whether or not the pointer to the
%       subscribed-to object is still valid.
% \item The \texttt{ConstraintMatrix} class gained a template parameter for the scalar
%       type and was been renamed \texttt{AffineConstraints}. Several methods that
%       take vectors or matrices as arguments,
%       such as \texttt{AffineConstraints::distribute\_local\_to\_global()},
%       now require that all matrix and vector arguments have matching number
%       types.
% \item Similarly, the functions \texttt{create\_mass\_matrix} and
%       \texttt{create\_boundary\_mass\_matrix} in the \texttt{MatrixCreator}
%       namespace no longer
%       support matrix and vector objects of different types.
\end{itemize}



%%%%%%%%%%%%%%%%%%%%%%%%%%%%%%%%%%%%%%%%%%%%%%%%%%%%%%%%%%%%%%%%%%%%%%%%%%%%%%%%
%%%%%%%%%%%%%%%%%%%%%%%%%%%%%%%%%%%%%%%%%%%%%%%%%%%%%%%%%%%%%%%%%%%%%%%%%%%%%%%%
%%%%%%%%%%%%%%%%%%%%%%%%%%%%%%%%%%%%%%%%%%%%%%%%%%%%%%%%%%%%%%%%%%%%%%%%%%%%%%%%
\section{How to cite \dealii}\label{sec:cite}

In order to justify the work the developers of \dealii put into this
software, we ask that papers using the library reference one of the
\dealii papers. This helps us justify the effort we put into it.

There are various ways to reference \dealii. To acknowledge the use of
the current version of the library, \textbf{please reference the present
document}. For up to date information and a bibtex entry for this document
see:
\begin{center}
 \url{https://www.dealii.org/publications.html}
\end{center}

The original \texttt{\dealii} paper containing an overview of its
architecture is \cite{BangerthHartmannKanschat2007}. If you rely on
specific features of the library, please consider citing any of the
following:
\begin{itemize}
 \item For geometric multigrid: \cite{Kanschat2004,JanssenKanschat2011,ClevengerHeisterKanschatKronbichler2019};
 \item For distributed parallel computing: \cite{BangerthBursteddeHeisterKronbichler11};
 \item For $hp$~adaptivity: \cite{BangerthKayserHerold2007};
  \item For partition-of-unity (PUM) and enrichment methods of the
    finite element space: \cite{Davydov2016};
 \item For matrix-free and fast assembly techniques:
   \cite{KronbichlerKormann2012,KronbichlerKormann2019};
 \item For computations on lower-dimensional manifolds:
   \cite{DeSimoneHeltaiManigrasso2009};
 \item For integration with CAD files and tools:
   \cite{HeltaiMola2015};
 \item For Boundary Elements Computations:
   \cite{GiulianiMolaHeltai-2018-a};
 \item For \texttt{LinearOperator} and \texttt{PackagedOperation} facilities:
   \cite{MaierBardelloniHeltai-2016-a,MaierBardelloniHeltai-2016-b}.
 \item For uses of the \texttt{WorkStream} interface:
   \cite{TKB16};
   \item For uses of the \texttt{ParameterAcceptor} concept, the
     \texttt{MeshWorker::ScratchData} base class, and the
     \texttt{ParsedConvergenceTable} class: \cite{SartoriGiulianiBardelloni-2018-a}.
\end{itemize}

\dealii can interface with many other libraries:
\begin{multicols}{3}
\begin{itemize}
\item ADOL-C \cite{Griewank1996a,adol-c}
\item ARPACK \cite{arpack}
\item Assimp \cite{assimp}
\item BLAS and LAPACK \cite{lapack}
\item cuSOLVER \cite{cusolver}
\item cuSPARSE \cite{cusparse}
\item Gmsh \cite{geuzaine2009gmsh}
\item GSL \cite{gsl2016}
\item Ginkgo \cite{ginkgo-web-page}
\item HDF5 \cite{hdf5}
\item METIS \cite{karypis1998fast}
\item MUMPS \cite{ADE00,MUMPS:1,MUMPS:2,mumps-web-page}
\item muparser \cite{muparser-web-page}
\item nanoflann \cite{nanoflann}
\item NetCDF \cite{rew1990netcdf}
\item OpenCASCADE \cite{opencascade-web-page}
\item p4est \cite{p4est}
\item PETSc \cite{petsc-user-ref,petsc-web-page}
\item ROL \cite{ridzal2014rapid}
\item ScaLAPACK \cite{slug}
\item SLEPc \cite{Hernandez:2005:SSF}
\item SUNDIALS \cite{sundials}
\item SymEngine \cite{symengine-web-page}
\item TBB \cite{Rei07}
\item Trilinos \cite{trilinos,trilinos-web-page}
\item UMFPACK \cite{umfpack}
\end{itemize}
\end{multicols}
Please consider citing the appropriate references if you use interfaces to these
libraries.

The two previous releases of \dealii can be cited as
\cite{dealII90,dealII91}.


\section{Acknowledgments}

\dealii is a world-wide project with dozens of contributors around the
globe. Other than the authors of this paper, the following people
contributed code to this release:\\
% updated 6/05/2019
  Giovanni Alzetta,
  Mathias Anselmann,
  Daniel Appel,
  Alexander Blank,
  Vishal Boddu,
  Benjamin Brands,
  Pi-Yueh Chuang,
  Sambit Das,
  Stefano Dominici,
  Nivesh Dommaraju,
  Niklas Fehn,
  Isuru Fernando,
  Andreas Fink,
  Rene Gassm{\"o}ller,
  Alexander Grayver,
  Joshua Hanophy,
  Logan Harbour,
  Daniel Jodlbauer,
  Stefan Kaessmair,
  Eldar Khattatov,
  Alexander Knieps,
  Uwe K{\"o}cher,
  Kurt Kremitzki,
  Dustin Kumor,
  Damien Lebrun-Grandie,
  Jonathan Matthews,
  Stefan Meggendorfer,
  Pratik Nayak,
  Lei Qiao,
  Ce Qin,
  Reza Rastak,
  Roland Richter,
  Alberto Sartori,
  Svenja Schoeder,
  Sebastian Stark,
  Antoni Vidal,
  Jiaxin Wang,
  Yuxiang Wang,
  Zhuoran Wang.

Their contributions are much appreciated!


\bigskip

\dealii and its developers are financially supported through a
variety of funding sources:

D.~Arndt and M.~Kronbichler were partially supported by the German
Research Foundation (DFG) under the project ``High-order discontinuous
Galerkin for the exa-scale'' (\mbox{ExaDG}) within the priority program ``Software
for Exascale Computing'' (SPPEXA).

W.~Bangerth, T.~C.~Clevenger, and T.~Heister were partially
supported by the National Science Foundation under award OAC-1835673
as part of the Cyberinfrastructure for Sustained Scientific Innovation (CSSI)
program  and by the Computational Infrastructure
in Geodynamics initiative (CIG), through the National Science
Foundation under Award No.~EAR-1550901 and The
University of California -- Davis.

W.~Bangerth and T.~Heister were also partially supported by award DMS-1821210.

D.~Davydov was supported by the German Research Foundation (DFG), grant DA
1664/2-1 and the Bayerisches Kompetenznetzwerk
f\"ur Technisch-Wissenschaftliches Hoch- und H\"ochstleistungsrechnen
(KONWIHR).

T.~Heister was also partially supported by NSF Award DMS-1522191, and
by Technical Data Analysis, Inc. through US Navy SBIR N16A-T003.

M.~Kronbichler was also supported by the Bayerisches Kompetenznetzwerk
f\"ur Technisch-Wissen\-schaft\-li\-ches Hoch- und H\"ochstleistungsrechnen
(KONWIHR) in the context of the project
``Performance tuning of high-order discontinuous Galerkin solvers for
SuperMUC-NG''.

R.~M.~Kynch was supported by the Engineering and Physical Science Research
Council (EPSRC) UK through grant EP/K023950/1 while working at Swansea
University, UK 2013-2015 where the majority of his contribution was
undertaken.

M.~Maier was partially supported by ARO MURI Award No. W911NF-14-0247.

B.~Turcksin: Research sponsored by the Laboratory Directed Research and
Development Program of Oak Ridge National Laboratory, managed by UT-Battelle,
LLC, for the U. S. Department of Energy.

D.~Wells was supported by the National Science Foundation (NSF) through Grant
DMS-1344962.

The Interdisciplinary Center for Scientific Computing (IWR) at Heidelberg
University has provided hosting services for the \dealii web page.


\bibliography{paper}{}
\bibliographystyle{abbrv}

\end{document}

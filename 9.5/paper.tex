\documentclass{ansarticle-preprint}
%\usepackage{ucs}
\usepackage[utf8]{inputenc}
\usepackage{amsmath}
%\usepackage{cite}
\usepackage{anslistings}
\usepackage{multicol}
\usepackage{pdfsync}
\usepackage{enumitem}

\usepackage{pgfplots}
\usepackage{pgfplotstable}

\usepackage{fontenc}
\usepackage{graphicx}
\usepackage{xspace}

\usepackage{siunitx}

\usepackage{floatflt}

\usepackage{multirow}

\usepackage{booktabs}

%\renewcommand{\baselinestretch}{2.0}	
%\usepackage{lineno}
%\renewcommand\linenumberfont{\normalfont\tiny}
%\linenumbers	

\graphicspath{{svg/}}

\usepackage[normalem]{ulem}

\usepackage{caption}
\usepackage{subcaption}

\usepackage{todonotes}

\pgfplotsset{compat=1.9}
\definecolor{gnuplot@lightblue}{RGB}{87,181,232}
\definecolor{gnuplot@green}{RGB}{0,158,115}
\definecolor{gnuplot@purple}{RGB}{148,0,212}

\newcommand{\specialword}[1]{\texttt{#1}}
\newcommand{\dealii}{{\specialword{deal.II}}\xspace}
\newcommand{\pfrst}{{\specialword{p4est}}\xspace}
\newcommand{\trilinos}{{\specialword{Trilinos}}\xspace}
\newcommand{\aspect}{\specialword{Aspect}\xspace}
\newcommand{\petsc}{\specialword{PETSc}\xspace}
\newcommand{\cmake}{{\specialword{CMake}}\xspace}
\newcommand{\candi}{{\specialword{candi}}\xspace}



\usetikzlibrary{shapes.misc}
\tikzset{cross/.style={cross out, draw=black, minimum size=2*(#1-\pgflinewidth), inner sep=0pt, outer sep=0pt},
%default radius will be 1pt.
cross/.default={2pt}}

%
% Author list -- please add yourself in both places below (in
%                alphabetical order) if you think that your
%                contributions to the last release warrant this
%

\hypersetup{
  pdfauthor={
    Daniel Arndt,
    Wolfgang Bangerth,
    Marc Fehling,
    Timo Heister,
    Luca Heltai,
    Martin Kronbichler,
    Matthias Maier,
    Peter Munch,
    Jean-Paul Pelteret,
    Bruno Turcksin,
    David Wells
  },
  pdftitle={The deal.II Library, Version 9.5, 2023},
}

\title{The \dealii Library, Version 9.5}

 \author[1*]{Daniel Arndt}
 \affil[1]{Scalable Algorithms and Coupled Physics Group,
   Computational Sciences and Engineering Division,
   Oak Ridge National Laboratory, 1 Bethel Valley Rd.,
   TN 37831, USA.
   \texttt{arndtd/turcksinbr@ornl.gov}}

 \author[2,3]{Wolfgang~Bangerth}
 \affil[2]{Department of Mathematics, Colorado State University, Fort
   Collins, CO 80523-1874, USA.
   \texttt{bangerth/marc.fehling@colostate.edu}}
 \affil[3]{Department of Geosciences, Colorado State University, Fort
   Collins, CO 80523, USA.}

\author[2]{Marc~Fehling}

\author[4]{Timo~Heister}
 \affil[4]{School of Mathematical and Statistical Sciences,
   Clemson University,
   Clemson, SC, 29634, USA
   {\texttt{heister@clemson.edu}}}

\author[5]{Luca~Heltai}
\affil[5]{SISSA,
   International School for Advanced Studies,
   Via Bonomea 265,
   34136, Trieste, Italy.
   {\texttt{marco.feder/luca.heltai@sissa.it}}}

 \author[6]{Martin~Kronbichler}
 \affil[6]{Institute of Mathematics,
   University of Augsburg,
   Universit\"atsstr.~12a, 86159 Augsburg, Germany.
   {\texttt{martin.kronbichler/peter.muench@uni-a.de}}}

\author[7]{Matthias~Maier}
\affil[7]{Department of Mathematics,
  Texas A\&M University,
  3368 TAMU,
  College Station, TX 77845, USA.
  {\texttt{maier@math.tamu.edu}}}

\author[6,8]{Peter Munch}
 \affil[8]{Institute of Material Systems Modeling,
 Helmholtz-Zentrum Hereon,
 Max-Planck-Str. 1, 21502 Geesthacht, Germany}


\author[9]{Jean-Paul~Pelteret}
\affil[9]{Independent researcher.
{\texttt{jppelteret@gmail.com}}}

\author[1*]{Bruno~Turcksin}

\author[10]{David Wells}
\affil[10]{Department of Mathematics, University of North Carolina,
  Chapel Hill, NC 27516, USA.
  {\texttt{drwells@email.unc.edu}}}

\renewcommand{\labelitemi}{--}


\begin{document}
\maketitle

\footnotetext{%
  $^\ast$ This manuscript has been authored by UT-Battelle, LLC under Contract No.
  DE-AC05-00OR22725 with the U.S. Department of Energy.
  %The United States
  %Government retains and the publisher, by accepting the article for
  %publication, acknowledges that the United States Government retains a
  %non-exclusive, paid-up, irrevocable, worldwide license to publish or reproduce
  %the published form of this manuscript, or allow others to do so, for United
  %States Government purposes. The Department of Energy will provide public
  %access to these results of federally sponsored research in accordance with the
  %DOE Public Access Plan (http://energy.gov/downloads/doe-public-access-plan).
}


\begin{abstract}
  This paper provides an overview of the new features of the finite element
  library \dealii, version 9.5.
\end{abstract}



%%%%%%%%%%%%%%%%%%%%%%%%%%%%%%%%%%%%%%%%%%%%%%%%%%%%%%%%%%%%%%%%%%%%%%%%%%%%%%%%
%%%%%%%%%%%%%%%%%%%%%%%%%%%%%%%%%%%%%%%%%%%%%%%%%%%%%%%%%%%%%%%%%%%%%%%%%%%%%%%%
%%%%%%%%%%%%%%%%%%%%%%%%%%%%%%%%%%%%%%%%%%%%%%%%%%%%%%%%%%%%%%%%%%%%%%%%%%%%%%%%
\section{Overview}

\dealii version 9.5.0 was released XX XX, 2023.
This paper provides an
overview of the new features of this release and serves as a citable
reference for the \dealii software library version 9.5. \dealii is an
object-oriented finite element library used around the world in the
development of finite element solvers. It is available for free under the
GNU Lesser General Public License (LGPL). Downloads are available at
\url{https://www.dealii.org/} and \url{https://github.com/dealii/dealii}.

The major changes of this release are:
%
\begin{itemize}
  \item Integration of \texttt{Kokkos} (see Section~\ref{sec:kokkos});
  \item Update to and extension of the \texttt{PETSc} wrappers (see Section~\ref{sec:petsc});
  \item Addition of new \texttt{Trilinos} wrappers (see Section~\ref{sec:trilinos});
  \item Advances in matrix-free infrastructure (see Section~\ref{sec:mf});
  \item Advances in non-matching support (see Section~\ref{sec:nonmatching});
  \item New features related to liner algebra (see Section~\ref{sec:lac})
  \item C++ language modernization (see Section~\ref{sec:language}).
\end{itemize}
%

While all of these major changes are discussed in detail in
Section~\ref{sec:major}, there
are a number of other noteworthy changes in the current \dealii release,
which we briefly outline in the remainder of this section:
%
\begin{itemize}
  \item The new function \texttt{CellAccessor::as\_dof\_handler\_iterator()}
  simplifies the conversion from a \texttt{Cell\-Accessor} to a \texttt{DoF\-Cell\-Accessor}.
  The old way 
\begin{c++}
const auto cell_dof = typename DoFHandler<dim, spacedim>::
  active_cell_iterator(&dof_handler.get_triangulation(),
    cell->level(), cell->index(), &dof_handler);
\end{c++}
was lengthy and error-prone. In contrast, the new way:
\begin{c++}
const auto cell_dof = cell->as_dof_handler_iterator(dof_handler);
\end{c++}
\end{itemize}
%
The changelog lists more than X other features and bugfixes.




%%%%%%%%%%%%%%%%%%%%%%%%%%%%%%%%%%%%%%%%%%%%%%%%%%%%%%%%%%%%%%%%%%%%%%%%%%%%%%%%
%%%%%%%%%%%%%%%%%%%%%%%%%%%%%%%%%%%%%%%%%%%%%%%%%%%%%%%%%%%%%%%%%%%%%%%%%%%%%%%%
%%%%%%%%%%%%%%%%%%%%%%%%%%%%%%%%%%%%%%%%%%%%%%%%%%%%%%%%%%%%%%%%%%%%%%%%%%%%%%%%
\section{Major changes to the library}
\label{sec:major}

This release of \dealii contains a number of large and significant changes,
which will be discussed in this section.
It of course also includes a
vast number of smaller changes and added functionality; the details of these
can be found
\href{https://dealii.org/developer/doxygen/deal.II/changes_between_9_4_2_and_9_5_0.html}
{in the file that lists all changes for this release}; see \cite{changes95}.

%\newpage

%%%%%%%%%%%%%%%%%%%%%%%%%%%%%%%%%%%%%%%%%%%%%%%%%%%%%%%%%%%%%%%%%%%%%%%%%%%%%%%%
\subsection{Integration of Kokkos}\label{sec:kokkos}



%%%%%%%%%%%%%%%%%%%%%%%%%%%%%%%%%%%%%%%%%%%%%%%%%%%%%%%%%%%%%%%%%%%%%%%%%%%%%%%%
\subsection{Update of and additions to the PETSc wrappers}\label{sec:petsc}

\begin{itemize}
\item PreconditionBDDC
\item SNES
\item TS
\item Wrapping PETSc matrices and vectors
\end{itemize}

%%%%%%%%%%%%%%%%%%%%%%%%%%%%%%%%%%%%%%%%%%%%%%%%%%%%%%%%%%%%%%%%%%%%%%%%%%%%%%%%
\subsection{New Trilinos wrappers}\label{sec:trilinos}

We have added a new wrapper to \texttt{Belos}. The package
\texttt{Belos}  is the
successor of \texttt{AztecOO} and provides basic and advanced iterative solvers
that heavily rely on multivector operations. The interface of the
wrapper is similar to the one of the native iterative solvers
provided by \dealii:

\begin{c++}
TrilinosWrappers::SolverBelos<VectorType> solver(/*...*/);
solver.solve(matrix, x, r, preconditioner);
\end{c++}

Via additional data, the user can select the actual iterative
solver to be used and set its configurations.

Furthermore, we have added a wrapper to \texttt{NOX}. This is a
nonlinear-solver library similar to \texttt{KINSOL} from \texttt{SUNDIALS}
and to \texttt{SNES} from \texttt{PETSc}. The basic interface
of the \texttt{NOX} wrapper is similar to the one of the other wrappers
mentioned before:

\begin{c++}
TrilinosWrappers::NOXSolver<VectorType> solver(/*...*/);

solver.residual            = [ ](const auto &src, auto &dst) {/*...*/};
solver.setup_jacobian      = [&](const auto &src) {/*...*/};
solver.apply_jacobian      = [&](const auto &src, auto &dst) {/*...*/};
solver.solve_with_jacobian = 
  [&](const auto &src, auto &dst, const auto tol) {/*...*/};

solver.solve(solution);
\end{c++}

We refer interested readers to our documentation for other more advanced
functions of this wrapper which are related to the differences in features of these libraries.
In particular, we have added the possibility to reuse the preconditioner between
nonlinear steps (also known as \textit{preconditioner lagging}), which is
unfortunately natively only  supported in the official \texttt{EPetra} implementations.


%%%%%%%%%%%%%%%%%%%%%%%%%%%%%%%%%%%%%%%%%%%%%%%%%%%%%%%%%%%%%%%%%%%%%%%%%%%%%%%%
\subsection{Advances in matrix-free infrastructure}\label{sec:mf}

In the matrix-free infrastructure, numerous advances have been made. Some of these
are:
\begin{itemize}
\item In release 9.3, we enabled parallel $hp$-operations in the matrix-free infrastructure.
The infrastructure did not work properly in the case that certain cells
did not get any degrees of freedom due to the usage of \texttt{FE\_Nothing}. This has been
fixed now. Furthermore, \texttt{FE\_Nothing} now also works together with DG (\texttt{FE\_DGQ}). Due to the popularity of \texttt{FE\_Nothing} as a mean to enable
or disable cells, we have introduced the new class
\texttt{ElementActivationAndDeactivationMatrixFree}, which wraps a \texttt{MatrixFree} object, only loops over all
active cells and optionally interprets faces between active and deactivated cells
as boundary face. A use case is shown in~\cite{proell2023highly} in the context of powder-bed-fusion additive
manufacturing.
\item The matrix-free infrastructure allows to interleave cell loops with vector updates
by providing \texttt{pre}/\texttt{post} functions that are run on index ranges. This
feature is used, e.g., in \dealii to improve the performance of (preconditioned)
conjugate gradient solvers~\cite{kronbichler2022cg} as well as of relaxation and Chebyshev iterations (see Subsection~\ref{sec:lac}).
Up to release~9.3, the \texttt{pre}/\texttt{post} infrastructure was only supported for
continuous elements (cell loop); now, it also works for discontinuous elements which require
face loops additionally.
\item The operator \texttt{CellwiseInverseMassMatrix} now also efficiently
evaluates the inverse for coupling (dyadic) coefficients in the case of  multiple
components:
\begin{align*}
\left(v_j, D_{ji} u_i  \right)_{\Omega^{(K)}}
\quad  1\le i,j \le c,
\end{align*}
with $c$ being the number of components and $D\in \mathbb{R}^{c\times c}$ the tensorial
coefficient.
This is possible due to the tensor-product structure of the resulting element matrix:
\begin{align*}
M =  ( I_1 \otimes N^T) (D \otimes I_2 ) ( I_1 \otimes N),
\end{align*}
with $N$ being the tabulated shape functions and $I_1$/$I_2$ the appropriate identity matrices.
For square and invertible $N$, the inverse is explicitly given as:
\begin{align*}
M^{-1} =  ( I_1 \otimes N^{-1}) (D^{-1} \otimes I_2 ) ( I_1 \otimes N^{-T}).
\end{align*}
Please note that $N^{-1} = N_{1D}^{-1} \otimes N_{1D}^{-1} \otimes N_{1D}^{-1}$ is given
for hypercube-shaped cells, allowing to use sum factorization.
\end{itemize}

%%%%%%%%%%%%%%%%%%%%%%%%%%%%%%%%%%%%%%%%%%%%%%%%%%%%%%%%%%%%%%%%%%%%%%%%%%%%%%%%
\subsection{Advances in non-matching support}\label{sec:nonmatching}

The (matrix-free) non-matching support of \texttt{deal.II} heavily relies
on the classes \texttt{FE\-Point\-Eval\-u\-ation} and \texttt{RemotePointEvaluation},
which have been introduced in release 9.3. While \texttt{FE\-Point\-Eval\-u\-a\-tion}
is responsible for efficient evaluation/integration at arbitrary (reference)
points within a cell, \texttt{RemotePointEvaluation} is responsible for
sorting the (real) points within cells and for the successive communication.

In the current release, we considerably optimized \texttt{FEPointEvaluation}, e.g,
by caching the evaluated shape functions, templating loop bounds, and
exploiting the tensor-product structure of the shape functions if all points are
positioned on a face, a common use case in the context of fluid-structure
interaction. Furthermore, the extended class \texttt{NonMatching::MappingInfo}
allows to precompute and store metric terms, like the Jacobian, its determinant,
or the normals. This is useful in cases in which these metric terms do not change
and can be reused, e.g., in the context of iterative solvers. This development
is part of the efforts to make the interfaces of the 
(matrix-free) non-matching support more similar to the ones of the 
established matrix-free infrastructure of \dealii for structured (quadrature) points.

In addition to these node-level performance optimizations, we added experimental
support 1) for generating intersections of distributed
non-matching grids and working on them
and 2) for multigrid with non-nested levels. In the following, we describe these
features in detail.

\subsubsection*{Intersected meshes}

\begin{itemize}
\item uses \texttt{CGAL}~\cite{cgal-user-ref}
\item data can be used to fill \texttt{RemotePointEvaluation} (communication-free)
\item used successfully in \cite{heinz2023high} to perform Nitsche-type mortaring in the context
of the conservative formulation of acoustic equations discretized with
DG, in order to suppress artificial modes
\end{itemize}

\subsubsection*{Non-nested multigrid}

The library \dealii provides traditionally strong support for multigrid methods.
We support, in the context of geometric multigrid, both local-smoothing~\cite{ClevengerHeisterKanschatKronbichler2019}
and global-coarsening algorithms~\cite{munch2022gc} for locally refined meshes. The global-coarsening
infrastructure, furthermore, allows to globally coarsen the polynomial degree ($p$-multigrid),
whose applicability to $hp$-adaptive problems has been demonstrated. Algebraic
multigrid is supported by interfacing to the external libraries \texttt{PETSc}
and \texttt{Trilinos}. In the current release, we have added support for the case
that multigrid levels are given by non-nested meshes~\cite{adams2002evaluation, bittencourt2001nonnested, bramble1991analysis}. An example for such meshes is
presented in Figure~\ref{fig:nonnested}.

\begin{figure}

\centering

\fbox{\begin{minipage}{0.31\textwidth}\centering 0\vspace{5cm}\end{minipage}}
\fbox{\begin{minipage}{0.31\textwidth}\centering 1\vspace{5cm}\end{minipage}}
\fbox{\begin{minipage}{0.31\textwidth}\centering 2\vspace{5cm}\end{minipage}}

\caption{Example of non-nested multigrid levels.}\label{fig:nonnested}

\end{figure}

The current implementation extends the existing global-coarsening infrastructure by
introducing, in addition to the (conformal) \texttt{MGTwoLevelTransfer},
 a new (non-conformal) two-level transfer operator

\begin{c++}
MGTwoLevelTransferNonNested two_level_transfer(/*...*/);

two_level_transfer.reinit(dof_handler_fine, dof_handler_coarse,
                          // the following parameters are optional
                          mapping_fine, mapping_coarse, 
                          constraints_fine, constraints_coarse)
\end{c++}

which can be passed to \texttt{MGTransferGlobalCoarsening} (and
\texttt{MGTransferBlockGlobalCoarsening}) in the usual way. Please note
the similarity between the interfaces of \texttt{MGTwoLevelTransfer} and
of \texttt{MGTwoLevel\-Transfer\-Non\-Nested}, with the difference that the latter also
takes \texttt{Mapping} instances. Furthermore, it is not limited to the case
that the \texttt{DoFHandler} instances need to share the same coarse triangulation
and the coarser mesh has to be able to be generated by a coarsening step from the fine
mesh.

At the time of writing, the \texttt{MGTwoLevelTransferNonNested} operator performs
a pointwise interpolation (injection) from the coarse mesh to the support points
of the fine mesh. Adopting the notation of~\cite{munch2022gc}, we perform
for prolongation
\begin{align*}
x^{(f)} = \mathcal{W}^{(f)} \circ \sum_{e \in \{\text{coarse cells}\}} \mathcal{S}_e^{(f)} \circ \mathcal{P}_e^{(f, c)}
\circ \mathcal{C}_e^{(c)} \circ \mathcal{G}_e^{(c)} x^{(c)},
\end{align*}
i.e., loop over all coarse cells and interpolate to all (fine support) points that fall
into the cell with \texttt{FEPointEvaluation}. The local result is scattered into a global
vector, which is finalized by a communication and a weighting step.

The current implementation works for scalar and vectorial continuous (\texttt{FE\_Q})
and discontinuous elements (\texttt{FE\_DGQ}) for hypercube-shaped cells. We plan to support
simplices and to add, in addition to the pointwise interpolation, also support
for projection. Furthermore, we envision to provide utility tools to generate coarser
meshes, based on a fine mesh. Currently, the users themselves 
have to provide such meshes,
e.g., by generating meshes with different cell sizes in external mesh-generation tools.

Note that the new class \texttt{MGTwoLevelTransferNonNested} is not limited to
multigrid but can be used to prolongate results and restrict residuals between any
two meshes. Indeed, the infrastructure has been successfully applied to perform
conservative interpolation between a fine (near) mesh and a coarse (far) mesh in the
context of aeroacoustic problems.


%%%%%%%%%%%%%%%%%%%%%%%%%%%%%%%%%%%%%%%%%%%%%%%%%%%%%%%%%%%%%%%%%%%%%%%%%%%%%%%%
\subsection{New features regarding linear algebra}\label{sec:lac}

In addition to the new wrappers for linear-algebra functionalities of \texttt{PETSc}
and \texttt{Trilinos} (see Sections~\ref{sec:petsc} and \ref{sec:trilinos}) as well
as the new non-nested multigrid infrastructure, 
we made multiple additions to our own linear-algebra infrastructure.

\begin{itemize}
  \item Both \texttt{SolverGMRES} and \texttt{SolverFGMRES} now support the classical
  Gram--Schmidt orthonormalization in addition to the existing modified one. This
  allows to reduce the cost of vector operations in terms of
  communication latency and memory transfer significantly.
  \item Our Chebyshev preconditioner (\texttt{PreconditionChebyshev}) now also
  supports James Lottes’s novel fourth-kind Chebyshev
  polynomial~\cite{lottes2022optimal, phillips2022optimal}.
  \item Our relaxation preconditioner (\texttt{PreconditionRelaxation}) now also
  allows to interleave cell loops and vector updates related to relaxation. The
  relaxation iteration reads as
  \begin{align*}
  x^{(i+1)} \gets x^{(i)} + \omega P^{-1}(b-Ax^{(i)}).
  \end{align*}
  E.g, in the case that the preconditioner $P$ is a diagonal matrix, the zeroing of
  the destination vector $x_{i+1}$ can be performed during a \texttt{pre}-operation  of $A$
  and the vector update $x^{(i+1)}_j \gets x^{(i)}_j + \omega P^{-1}_{j,j}(b_j-(Ax^{(i)})_j)$
  during a \texttt{post} operation, allowing to reduce the number of read and write
  accesses from 8 and 4 to 1 and 3, respectively. The existing \texttt{pre}/\texttt{post}
  support in our Chebyshev-preconditioner implementation (\texttt{PreconditionChebyshev})
  has been improved. In addition, both \texttt{PreconditionRelaxation} and 
  \texttt{PreconditionChebyshev} support \texttt{pre}/\texttt{post} optimizations
  now not only for diagonal preconditioners but also for preconditioners that are
  built around cell loops and, as a consequence, support interleaving. An example of
  such a preconditioner are patch-based additive Schwarz preconditioners.
  \item In the context of additive Schwarz methods, the preconditioner application
  is defined as
  \begin{align*}
  v = P^{-1} u = \sum R_i^\top A_i^{-1} R_i u 
  \end{align*}
  with $A_i = R_i A R_i^T$ being a block of the assembled system matrix restricted
  to an index set (inverse of the assembly step). During the restriction step,
  rows of the system matrix that are potentially owned by other
  processes are needed. In \texttt{deal.II}, it is not possible to access remote entries
  of sparse matrices. Two novel functions allow to query this information. The
  function \texttt{restrict\_to\_serial\_sparse\_matrix()} creates, based
  on a given index set, a serial
  sparse matrix from a distributed one on each process:
  
\begin{c++}
SparseMatrixTools::restrict_to_serial_sparse_matrix (
  sparse_matrix_in, sparsity_pattern, requested_is, 
  system_matrix_out, sparsity_pattern_out)
\end{c++}

This function can be used, e.g., if the granularity of the additive Schwarz preconditioner
is a complete subdomain, potentially, with a fixed overlap.

In contrast, \texttt{restrict\_to\_full\_matrices()} performs the restriction
for arbitrary number of patches/blocks:

\begin{c++}
SparseMatrixTools::restrict_to_full_matrices (
  sparse_matrix_in, sparsity_pattern, indices_of_blocks, blocks)
\end{c++}

The data is stored in full matrices, since the typical granularity is a (rather small) cell-centric or
vertex-star patch.
  
  \item For certain types of configurations, there are computationally more efficient
  approaches than extracting submatrices from an assembled matrix. For example, the Laplace
  operator on 2D Cartesian meshes, the (element/patch) matrix is given as
  \begin{align*}
  A_i^{\text{cart}} = K_1 \otimes M_0 + M_1 \otimes K_0,
  \end{align*}
  i.e., as the tensor product of 1D mass and stiffness matrices. The inverse is
  explicitly given according to the fast diagnilization method~\cite{lynch1964direct}, as
  \begin{align*}
  \left(A_i^{\text{cart}}\right)^{-1} = (T_1 \otimes T_0) (\Lambda_1 \otimes I + I \otimes \Lambda_0)^{-1} (T_1^\top \otimes T_0^\top),
  \end{align*}
  with $T_i$ and $\Lambda_i$, being the (orthonormal) eigenvectors and the diagonal
  matrix of eigenvalues, obtained from a generalized eigendecomposition
  $K_iT_i = \Lambda_i M_i T_i$. Since $A_i \approx A_i^{\text{cart}}$
  might be a good approximation also in the case of non-Cartesian meshes and 
  $A_i^{\text{cart}}$ has an explicit inverse, it is considered in the
  literature as patch preconditioners in the context of additive
  Schwarz~\cite{witte2021fast, phillips2021auto, couzy1995spectral} and block-Jacobi methods~\cite{kronbichler2019hermite}.
  In \dealii, the new function
  \texttt{Tensor\-Product\-Matrix\-Creator::create\_\allowbreak laplace\_\allowbreak tensor\_\allowbreak product\_\allowbreak matrix()} computes $T_i$ and $\Lambda_i$
  for cell-centric patches with a specified overlap and given boundary conditions.
  A set of $T_i$ and $\Lambda_i$ is applied to a cell via
  \texttt{Tensor\-Product\-Matrix\-Symmetric\-Sum} or to a collection of cells
  via the new class \texttt{Tensor\-Product\-Matrix\-Symmetric\-Sum\-Collection}, which
  tries to reuse the eigenvalues and eigenvectors between cells.
\end{itemize}


%%%%%%%%%%%%%%%%%%%%%%%%%%%%%%%%%%%%%%%%%%%%%%%%%%%%%%%%%%%%%%%%%%%%%%%%%%%%%%%%
\subsection{C++ language modernization}\label{sec:language}

Mention the C++20 work. Starting point: \dealii{} relies heavily on
templates that are explicitly instantiated.


%%%%%%%%%%%%%%%%%%%%%%%%%%%%%%%%%%%%%%%%%%%%%%%%%%%%%%%%%%%%%%%%%%%%%%%%%%%%%%%%
\subsection{Build-system modernization}\label{sec:buildsystem}

?


%%%%%%%%%%%%%%%%%%%%%%%%%%%%%%%%%%%%%%%%%%%%%%%%%%%%%%%%%%%%%%%%%%%%%%%%%%%%%%%%
\subsection{New and improved tutorials and code gallery programs}
\label{subsec:steps}

Many of the \dealii tutorial programs were revised in a variety of ways
as part of this release. 
%In addition, there are a number of new tutorial
%programs:
%\begin{itemize}
%  \item
%\end{itemize}

There is also two new programs in the code gallery (a collection of
user-contributed programs that often solve more complicated problems
than tutorial programs, and that are intended as starting points for further
research rather than as teaching tools):
\begin{itemize}
  \item The program \texttt{A posteriori error estimator for first order hyperbolic problems}
  was contributed by Marco Feder.
  \item The program \texttt{Distributed moving laser heating} was contributed by
  Hongfeng Ma and Tatiana E. Itina.
\end{itemize}



%%%%%%%%%%%%%%%%%%%%%%%%%%%%%%%%%%%%%%%%%%%%%%%%%%%%%%%%%%%%%%%%%%%%%%%%%%%%%%%%
\subsection{Incompatible changes}\label{subsec:deprecated}

The 9.5 release includes
\href{https://dealii.org/developer/doxygen/deal.II/changes_between_9_4_2_and_9_5_0.html}
{around X incompatible changes}; see \cite{changes95}. The majority of these changes
should not be visible to typical user codes; some remove previously
deprecated classes and functions; and the majority changes internal
interfaces that are not usually used in external
applications. That said, the following are worth mentioning since they
may have been more widely used:
\begin{itemize}
  \item 
\end{itemize}



%%%%%%%%%%%%%%%%%%%%%%%%%%%%%%%%%%%%%%%%%%%%%%%%%%%%%%%%%%%%%%%%%%%%%%%%%%%%%%%%
%%%%%%%%%%%%%%%%%%%%%%%%%%%%%%%%%%%%%%%%%%%%%%%%%%%%%%%%%%%%%%%%%%%%%%%%%%%%%%%%
%%%%%%%%%%%%%%%%%%%%%%%%%%%%%%%%%%%%%%%%%%%%%%%%%%%%%%%%%%%%%%%%%%%%%%%%%%%%%%%%
\section{How to cite \dealii}\label{sec:cite}

In order to justify the work the developers of \dealii put into this
software, we ask that papers using the library reference one of the
\dealii papers. This helps us justify the effort we put into this library.

There are various ways to reference \dealii. To acknowledge the use of
the current version of the library, \textbf{please reference the present
  document}. For up-to-date information and a bibtex entry
see
\begin{center}
  \url{https://www.dealii.org/publications.html}
\end{center}

The original \dealii paper containing an overview of its
architecture is \cite{BangerthHartmannKanschat2007}, and a more recent
publication documenting \dealii's design decisions is available as \cite{dealII2020design}. If you rely on
specific features of the library, please consider citing any of the
following:
\begin{multicols}{2}
  \vspace*{-36pt}
  \begin{itemize}[leftmargin=4mm]
    \item For geometric multigrid: \cite{Kanschat2004,JanssenKanschat2011,ClevengerHeisterKanschatKronbichler2019, munch2022gc};
    \item For distributed parallel computing: \cite{BangerthBursteddeHeisterKronbichler11};
    \item For $hp$-adaptivity: \cite{BangerthKayserHerold2007,fehling2022};
    \item For partition-of-unity (PUM) and finite element enrichment methods:
           \cite{Davydov2016};
    \item For matrix-free and fast assembly techniques:
          \cite{KronbichlerKormann2012,KronbichlerKormann2019};
    \item For computations on lower-dimensional manifolds:
          \cite{DeSimoneHeltaiManigrasso2009};
    \item For curved geometry representations and manifolds:
          \cite{HeltaiBangerthKronbichlerMola2019};
    \item For integration with CAD files and tools:
          \cite{HeltaiMola2015};
    \item For boundary element computations:
          \cite{GiulianiMolaHeltai-2018-a};
    \item For the \texttt{LinearOperator} and
      \texttt{Packaged\-Operation} facilities:
          \cite{MaierBardelloniHeltai-2016-a,MaierBardelloniHeltai-2016-b};
    \item For uses of the \texttt{WorkStream} interface:
          \cite{TKB16};
    \item For uses of the \texttt{ParameterAcceptor} concept, the
          \texttt{MeshWorker::ScratchData} base class, and the
          \texttt{ParsedConvergenceTable} class:
          \cite{SartoriGiulianiBardelloni-2018-a};
    \item For uses of the particle functionality in \dealii:
          \cite{GLHPB18}.
          \vfill\null
  \end{itemize}
\end{multicols}

\dealii can interface with many other libraries:
\begin{multicols}{3}
  \begin{itemize}[leftmargin=4mm]
    \item ADOL-C \cite{griewank1996adolc}
    \item ArborX \cite{lebrun2020arborx}
    \item ARPACK \cite{lehoucq1998arpack}
    \item Assimp \cite{schulze2021assimp}
    \item BLAS and LAPACK \cite{anderson1999lapack}
    \item Boost \cite{boost-web-page}
    \item CGAL \cite{cgal-user-ref}
    \item cuSOLVER \cite{cusolver-web-page}
    \item cuSPARSE \cite{cusparse-web-page}
    \item Gmsh \cite{geuzaine2009gmsh}
    \item GSL \cite{galassi2009gsl,gsl-web-page}
    \item Ginkgo \cite{anzt2020ginkgo,anzt2022ginkgo}
    \item HDF5 \cite{hdf5-web-page}
    \item METIS \cite{karypis1998metis}
    \item MUMPS \cite{amestoy2001mumps,amestoy2019mumps}
    \item muparser \cite{muparser-web-page}
    \item OpenCASCADE \cite{opencascade-web-page}
    \item p4est \cite{burstedde2011p4est,burstedde2020parallel}
    \item PETSc \cite{petsc-user-ref,petsc-web-page}
    \item ROL \cite{ridzal2014rol}
    \item ScaLAPACK \cite{blackford1997scalapack}
    \item SLEPc \cite{hernandez2005slepc}
    \item SUNDIALS \cite{hindmarsh2005sundials}
    \item SymEngine \cite{symengine-web-page}
    \item TBB \cite{reinders2007tbb}
    \item Trilinos \cite{heroux2005trilinos,trilinos-web-page}
    \item UMFPACK \cite{davis2004umfpack}
  \end{itemize}
\end{multicols}
Please consider citing the appropriate references if you use
interfaces to these libraries.

The two previous releases of \dealii can be cited as
\cite{dealII92,dealII93}.


\section{Acknowledgments}

\dealii is a worldwide project with dozens of contributors around the
globe. Other than the authors of this paper, the following people
contributed code to this release:\\
%
% Uwe Koecher doesn't usually show up in the changelog, but
% we should make sure he's listed.
%

% This is up-to-date as of 9.4 RC1 - should be the final list.
.


Their contributions are much appreciated!


\bigskip

\dealii and its developers are financially supported through a
variety of funding sources:


D.~Arndt and B.~Turcksin: Research sponsored by the Laboratory Directed Research and
Development Program of Oak Ridge National Laboratory, managed by UT-Battelle,
LLC, for the U. S. Department of Energy.

W.~Bangerth, T.~Heister, and R.~Gassm{\"o}ller were partially
supported by the Computational Infrastructure for Geodynamics initiative
(CIG), through the National Science Foundation (NSF) under Award
No.~EAR-1550901 and The University of California -- Davis.

W.~Bangerth and M.~Fehling were partially supported by Award OAC-1835673
as part of the Cyberinfrastructure for Sustained Scientific Innovation (CSSI)
program.

W.~Bangerth was also partially supported by Awards DMS-1821210 and EAR-1925595.

T.~Heister was also partially supported by NSF
Awards OAC-2015848, DMS-2028346, and
EAR-1925575.

L.~Heltai was partially supported by the Italian Ministry of Instruction,
University and Research (MIUR), under the 2017 PRIN project NA-FROM-PDEs MIUR
PE1, ``Numerical Analysis for Full and Reduced Order Methods for the efficient
and accurate solution of complex systems governed by Partial Differential
Equations''.

M.~Kronbichler and P.~Munch were partially supported by the
Bayerisches Kompetenznetzwerk
f\"ur Technisch-Wissen\-schaft\-li\-ches Hoch- und H\"ochstleistungsrechnen
(KONWIHR) in the context of the projects
``High-order matrix-free finite element implementations with
hybrid parallelization and improved data locality'' and ``Fast and scalable finite element algorithms for coupled multiphysics problems and non-matching grids''.

M.~Maier was partially supported by NSF Awards DMS-1912847 and DMS-2045636.

D.~Wells was supported by the NSF Award OAC-1931516.

The Interdisciplinary Center for Scientific Computing (IWR) at Heidelberg
University has provided hosting services for the \dealii web page.

\bibliography{paper}{}
\bibliographystyle{abbrv}

\end{document}

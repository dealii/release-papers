\documentclass{ansarticle}
%\usepackage{ucs}
\usepackage[utf8]{inputenc}
\usepackage{amsmath}
%\usepackage{cite}
\usepackage{anslistings}

\usepackage{fontenc}
\usepackage{graphicx}

\hypersetup{
  pdfauthor={Wolfgang Bangerth, Timo Heister, Luca Heltai, Guido Kanschat, Martin Kronbichler, Matthias Maier, Bruno Turcksin, Toby D. Young},
  pdftitle={The deal.II Library, Version 8.2, 2015},
}

\newcommand{\specialword}[1]{\texttt{#1}}
\newcommand{\dealii}{{\specialword{deal.II}}}
\newcommand{\pfrst}{{\specialword{p4est}}}
\newcommand{\trilinos}{{\specialword{Trilinos}}}
\newcommand{\aspect}{\specialword{Aspect}}
\newcommand{\petsc}{\specialword{PETSc}}
\newcommand{\cmake}{{\specialword{CMake}}}
\newcommand{\autoconf}{{\specialword{autoconf}}}

\title{The \dealii{} Library, Version 8.2}
\author[1]{Wolfgang Bangerth}
\affil[1]{Department of Mathematics, Texas A\&M University, College Station, 
    TX 77843, USA,
    %\href{mailto:bangerth@math.tamu.edu}
    {\texttt{bangerth@math.tamu.edu}}}
\author[2]{Timo Heister}
\affil[2]{Mathematical Sciences,
O-110 Martin Hall.
Clemson University.
Clemson, SC 29634, USA,
%\href{mailto:heister@clemson.edu}
{\texttt{heister@clemson.edu}}}
\author[3]{Luca Heltai}
\affil[3]{SISSA - International School for Advanced Studies, Via
  Bonomea 265, 34136 Trieste, Italy,
  %\href{mailto:luca.heltai@sissa.it}
  {\texttt{luca.heltai@sissa.it}}}
\author[4]{Guido Kanschat}
\affil[4]{Interdisciplinary Center for Scientific Computing (IWR),
  Universität Heidelberg, Im Neuenheimer Feld 368, 69120 Heidelberg, Germany,
  %\href{mailto:kanschat@uni-heidelberg.de}
  {\texttt{kanschat@uni-heidelberg.de}}}
\author[5]{Martin Kronbichler}
\affil[5]{Institute for Computational Mechanics, Technische
  Universität München, Boltzmannstr.~15, 85748 Garching b. München,
  Germany,
  %\href{mailto:kronbichler@lnm.mw.tum.de}
  {\texttt{kronbichler@lnm.mw.tum.de}}}
\author[6]{Matthias Maier}
\affil[6]{Institute of Applied Mathematics, Heidelberg
  University, Im Neuenheimer Feld 293/294, 69120 Heidelberg, Germany,
  %\href{mailto:matthias.maier@iwr.uni-heidelberg.de}
  {\texttt{matthias.maier@iwr.uni-heidelberg.de}}}
\author[7]{Bruno Turcksin}
\affil[7]{Department of Mathematics, Texas A\&M University, College Station, 
  TX 77843, USA,
  %\href{mailto:turcksin@math.tamu.edu}
  {\texttt{turcksin@math.tamu.edu}}}
\author[8]{Toby~D.~Young}
\affil[8]{Institute of Fundamental Technological Research of the Polish Academy of Sciences, ul. Pawińskiego 5b, Warsaw 02-106, Poland,
  %\href{mailto:tyoung@ippt.pan.pl}
  {\texttt{tyoung@ippt.pan.pl}}}

\renewcommand{\labelitemi}{--}


\begin{document}
\maketitle

\begin{abstract}
  This paper provides an overview of the new features of the finite element library \dealii{} version 8.2.
\end{abstract}


\section{Overview}

\dealii{} version 8.2 was released January 1, 2015. This paper provides an
overview of the new features of this release and serves as a citable
reference for the \dealii{} software library version 8.2. \dealii{} is an
object-oriented finite element library used around the world in the
development of finite element solvers. It is available for free under the
GNU Lesser General Public License (LGPL) from the \dealii{} homepage at
\url{http://www.dealii.org/}.


Version 8.2 contains,  along
with the usual set of new functions, bug fixes and documentation updates, the following noteworthy changes:
\begin{itemize}
\item Comprehensive support for geometries described by arbitrary manifolds
  and meshes that respect this description not only on the boundary but
  also internally
\item Support for geometries imported from CAD using the OpenCASCADE library
\item Three new tutorial programs on complex geometries, CAD geometries, and
  time stepping methods
\item Support for users wanting to use C++11 features
\item Improvements to multithreading support
\item Vectorization of many vector operations using OpenMP SIMD directives
\end{itemize}
Some of these will be detailed in the following section.
Information on how to cite \dealii{} is provided in Section \ref{sec:cite}.


\section{Significant changes to the library}

This release of deal.II contains a number of large and significant changes
that will be discussed in the following sections. It of course also contains a
vast number of smaller changes and added functionality; the details of these
can be found in the file that lists all changes for this release and that is
linked to from the web site of each release as well as the release
announcement.


\subsection{Supporting complex geometries through manifolds}

Complex geometries can be described using the concept of a manifold, a
topological space that resembles Euclidean space near each point. In
mathematics, particularly topology, one describes a manifold using an
atlas. An atlas consists of individual charts that, roughly speaking,
describe individual regions of the manifold, by means of providing a
local coordinate system near each point. (Such coordinate systems may not
exist \textit{globally}, as in the case of the surface of a sphere, but it
must be possible to cover the entire manifold with charts that are valid in
some local neighborhood of every point.) In \dealii{} these concepts
have been formalized in the abstract base class \verb|Manifold| that
defines an interface by which the \verb|Triangulation| and
\verb|Mapping| classes can query geometric information about the
domain.

Whenever a new point is required (for example upon mesh refinement, or
when integrating over curved manifolds using higher order mappings), new points are typically
obtained by providing a local coordinate system on the manifold,
identifying existing points in the local coordinate system (pulling
them back using the local chart to obtain their local coordinates),
finding the new point in the local coordinate system by weighted sums of
the existing points, and transforming the point back to the real space
(pushing it forward using the local chart).

A new identifier (the \verb|manifold_id|) has been introduced to
describe geometrical features of complex domains: every entity of a mesh
(i.e., cells, faces, and edges) now have a \verb|manifold_id| associated with
them. This extends the existing behavior: Previous releases of
the library allowed only boundaries to be curved, and used the
\verb|boundary_id| of the faces of a triangulation to identify how to
add new points upon mesh refinement.  In contrast, the new \verb|manifold_id|
also exists for cells and interior faces. Upon refinement, the
\verb|Triangulation| class queries each object (hex, quad or line)
where new points should be placed. Attaching a custom \verb|Manifold|
to any such object ensures that new points are created consistently
with the underlying geometry.

We provide a series of classes that implement this mechanism for
commonly used geometries:
\begin{itemize}
\item \verb|FlatManifold|: in the simplest case, the
  objects that make up the \verb|Triangulation| are straight line
  segments, a bi-linear surface or a tri-linear volume. New vertices
  are then simply put into the middle of the old ones (where
  ``middle'' means a suitable average of the locations of the
  pre-existing vertices). This is the default manifold associated to
  each object of the \verb|Triangulation|;
\item \verb|SphericalManifold|: you can use this manifold object to
  describe any sphere, circle, hypersphere or hyperdisc in two or
  three dimensions, both as a co-dimension one manifold descriptor or
  as co-dimension zero manifold descriptor;
\item \verb|CylindricalManifold|: in three dimensions, points are
  transformed using a cylindrical coordinate system along an
  arbitrarily oriented cylinder.
\end{itemize}

More general specialisations can be obtained by using the
\verb|ChartManifold| partial specialisation: this class is intended to
directly implement the topological concept of a chart, by explicitly
providing \emph{pull back} and \emph{push forward} expressions. Classes
derived from \verb|ChartManifold| only have to overload the \verb|pull_back|
and \verb|push_forward| functions, while all other functions are then defined
in terms of these two functions. \verb|ChartManifold| assumes that the
manifold can be represented by a single, globally valid chart.
The
class \verb|FunctionManifold| further implements this using explicit \dealii{}
\verb|Function| classes, and the new tutorial \verb|step-53| provides a
concrete example on how to use these classes to work with complex
geometries.

\subsection{Interfacing with CAD geometries}

If you have OpenCASCADE (\url{http://www.opencascade.org/})
installed\cite{opencascade-web-page}, you can now use existing IGES or
STEP CAD files to describe the boundary of your geometry to \dealii{}.

The \dealii{} library allows the user to specify the geometry of the
analysis in one of two possible ways: i) by creating a coarse grid
internally (for simple geometries, such as boxes, cylinders, shells,
etc.) or ii) by reading a mesh input file in one of the supported
formats via the \verb|GridIn| class.

Along with the default boundary descriptor \verb|StraightBoundary|
which implements the standard behavior of \dealii{}, some elementary
boundary descriptors are also included in the library to describe
circles or spheres (\verb|HyperBallBoundary|), cylinders
(\verb|CylinderBoundary|), half circles or half spheres
(\verb|HalfHyperBallBoundary|), and so on. In general, \dealii{}
provides a curved boundary descriptor for each one of the elementary
mesh that can be created from within the library itself.

Real world geometries, however, can rarely be described by straight
lines, balls or cylinders, and much more complex structures are
required if one wants to tackle industrial problems. The industry
standard for the design of any geometry is based on Computer Aided
Design (CAD) tools, which rely on Non Uniform Rational B-Splines
(NURBS) descriptions of curves and surfaces. Virtually any object that
surrounds us has been modeled using some NURBS patch in a CAD tool.

The namespace \verb|dealii::OpenCASCADE| contains all utilities and
wrappers that are needed to manipulate CAD files. Most of the
functions in the \verb|dealii::OpenCASCADE| namespace deal with
\verb|TopoDS_Shape| objects: the default topological object of the
\verb|OpenCASCADE| library, and provide interfaces to deal.II objects,
like \verb|Triangulation|, \verb|Manifold| and \verb|Boundary|.

The new step-54 tutorial program of the \dealii{} library shows an example
application of the functions and classes in the
\verb|deal.II::OpenCASCADE| namespace. In step-54, a CAD shape and a
coarse surface triangulation are imported from external files, and the
resulting triangulation is then refined on the CAD surface using
different projection strategies. A detailed description of these
techniques is available in~\cite{HeltaiMola2015}.

\subsection{A new tutorial: Support for time stepping methods (step-52)}
The goal of this tutorial is to show how to use the different Runge--Kutta
methods of the new TimeStepping class. The Runge--Kutta methods used in this
tutorial include explicit methods, implicit methods, and embedded methods. In
step-52, the time dependent neutron diffusion equation is solved on a uniform
mesh. To compare the different time stepping methods, the solution of the
problem is chosen such that its spatial component can be exactly represented on
the uniform mesh by quadratic continuous finite elements. Moreover, at the
end of the last time step the exact solution is zero. Therefore, the different
time discretization methods can be compared by simply looking at the discretized
solutions.

\subsection{Support for some C++11 features}
The C++11 standard \cite{cpp11} provides many convenient features that can
help make writing finite element codes simpler. \dealii{} does not use any of
these features itself to ensure that it can be used with compilers that do not
implement C++11 yet. 

However, this does not prevent users from using C++11
features in their own codes. Among these, range-based for loops are
particularly useful given that \dealii{} programs often contain many loops
over all cells and that writing these loops is frequently quite verbose:
\begin{c++}
  Triangulation<dim> triangulation;
  ...
  typename Triangulation<dim>::active_cell_iterator
    cell = triangulation.begin_active(),
    endc = triangulation.end();
  for (; cell!=endc; ++cell)
    cell->set_refine_flag();
\end{c++}
Code such as this can be significantly simplified by combining C++11's
range-based for loops and \texttt{auto}-typed variables. To support this,
\dealii{}'s container classes (i.e., the triangulations and DoF handler
classes) provide member functions that return objects that denote the entire
set of cells, active cells, or cells on a particular level. This allows to
write the same loop as above in the following, much shorter way:
\begin{c++}
  Triangulation<dim> triangulation;
  ...
  for (auto cell : triangulation.active_cell_iterators())
    cell->set_refine_flag();
\end{c++}
The function
\texttt{Triangulation::active\_cell\_iterators()}, 
and equivalents in the
\texttt{DoFHandler} and
\texttt{hp::DoFHandler} classes, were added in \dealii{} 8.2
to support this idiom.
Other variants of these functions provide iterator ranges
for all cells (not just the active ones) and for cells on individual
levels. A new documentation module elaborates on support for C++11 in
\dealii{}.


\section{How to cite \dealii{}}\label{sec:cite}

In order to justify the work the developers of \dealii{} put into this
software, we ask that papers using the library reference one of the \dealii{}
papers. This helps us justify the effort we put into it.

There are various ways to reference \dealii{}. To acknowledge the use of a
particular version of the library, reference the present document. For
up to date information and bibtex snippets for this document see:

\bigskip

\begin{center}
 \url{https://www.dealii.org/publications.html}
\end{center}

\bigskip

% \begin{minipage}{0.9\textwidth}%no page break in here please
% \begin{verbatim}
% @article{dealII82,
%   title = {The {\tt deal.{I}{I}} Library, Version 8.2},
%   author = {W. Bangerth and T. Heister and L. Heltai 
%    and G. Kanschat and M. Kronbichler and M. Maier
%    and B. Turcksin and T. D. Young},
%   journal = {preprint},
%   year = {2015},
% }
% \end{verbatim} %  journal = {arXiv preprint \url{http://arxiv.org/abs/TODO}},
% \end{minipage}

The original \texttt{\dealii{}} paper containing an overview of its
architecture is \cite{BangerthHartmannKanschat2007}. If you rely on specific
features of the library, please consider citing any of the following:
\begin{itemize}
 \item For geometric multigrid: \cite{Kanschat2004,JanssenKanschat2011};
 \item For distributed parallel computing: \cite{BangerthBursteddeHeisterKronbichler11};
 \item For $hp$ adaptivity: \cite{BangerthKayserHerold2007};
 \item For matrix-free and fast assembly techniques:
   \cite{KronbichlerKormann2012};
 \item For computations on lower-dimensional manifolds:
   \cite{DeSimoneHeltaiManigrasso2009};
 \item For integration with CAD files and tools:
   \cite{HeltaiMola2015}.
\end{itemize}

\dealii{} can interface with many other libraries:
\begin{itemize}
\item ARPACK \cite{arpack}
\item BLAS, LAPACK
\item HDF5 \cite{hdf5}
\item METIS \cite{karypis1998fast}
\item MUMPS \cite{ADE00,MUMPS:1,MUMPS:2,mumps-web-page}
\item muparser \cite{muparser-web-page}
\item NetCDF \cite{rew1990netcdf}
\item OpenCASCADE \cite{opencascade-web-page}
\item p4est \cite{p4est}
\item PETSc \cite{petsc-user-ref,petsc-web-page}
\item SLEPc \cite{Hernandez:2005:SSF}
\item Threading Building Blocks \cite{Rei07}
\item Trilinos \cite{trilinos,trilinos-web-page}
\item UMFPACK \cite{umfpack}
\end{itemize}
Please consider citing the appropriate references if you use interfaces to these
libraries.

Older releases of \dealii{} can be cited as \cite{dealII80,dealII81}.

\nocite{BangerthKanschat1999}

\section{Acknowledgements}

\dealii{} is a world-wide project with dozens of contributors around the
globe. Other than the authors of this paper, the following people contributed code to
this release:
Alexander Grayver,
Andrea Mola,
Andrew Baker,
Angela Klewinghaus,
Arezou Ghesmati,
Ben Thompson,
Christoph Heiniger,
Damien Lebrun-Grandie,
Daniel Arndt,
David Wells,
Denis Davydov,
Fahad Alrashed,
Giorgos Kourakos,
Jean-Paul Pelteret,
Kainan Wang,
Kevin Drzycimski,
Krysztof Bzowski,
Lukas Korous,
Manuel Quezada de Luna,
Markus Bürg,
Martin Steigemann,
Mayank Sabharwal,
Michal Wichrowski,
Mihai Alexe,
Minh Do-Quang,
Shiva Rudraraju,
Uwe Köcher,
Valentin Zingan,
Their contributions are much appreciated!


\dealii{} and its developers are financially supported through a
variety of funding sources. W.~Bangerth and B.~Turcksin were partially
supported by the National Science Foundation under award OCI-1148116
as part of the Software Infrastructure for Sustained Innovation (SI2)
program; by the Computational Infrastructure in Geodynamics initiative
(CIG), through the National Science Foundation under Award
No.~EAR-0949446 and The University of California -- Davis; and through
Award No.~KUS-C1-016-04, made by King Abdullah University of Science
and Technology (KAUST). 

L.~Heltai was partially supported by the project OpenViewSHIP,
``Sviluppo di un ecosistema computazionale per la progettazione
idrodinamica del sistema elica-carena'', financed by Regione FVG - PAR
FSC 2007-2013, Fondo per lo Sviluppo e la Coesione.

T.~Heister was partially supported by the Computational Infrastructure in
Geodynamics initiative (CIG), through the National Science Foundation under Award
No. EAR-0949446 and The University of California -- Davis and through Award No. KUS-
C1-016-04, made by King Abdullah University of Science and Technology (KAUST).

The Interdisciplinary Center for Scientific Computing (IWR) at Heidelberg University has provided 
hosting services for the deal.II web page and the SVN archive.
 
%\marginpar{Everyone's funding, please}






\bibliography{deal82}{}
\bibliographystyle{plain}

\end{document}

%%% Local Variables:
%%% mode: latex
%%% TeX-master: t
%%% End:
